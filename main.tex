\documentclass[master]{thuthesis}

% 所有其它可能用到的包都统一放到这里了,可以根据自己的实际添加或者删除。
\usepackage{thutils}
% 自己添加的包,用于使用listings
\usepackage{xcolor}
\usepackage{listings}
% 自己添加的包,用于排版算法
\usepackage{algorithm}
%\usepackage{algorithmic}
\usepackage{algpseudocode}
\usepackage{amssymb}
\usepackage{CJKfntef}

% 自己添加的格式定义,用于使用listings包排版Unix命令
\lstdefinestyle{BashInputStyle}{
	language=bash,
	basicstyle=\small\sffamily,
	numbers=left,
	numberstyle=\tiny,
	numbersep=3pt,
	frame=tb,
	columns=fullflexible,
	backgroundcolor=\color{yellow!20},
	linewidth=0.9\linewidth,
	xleftmargin=0.1\linewidth
}

\begin{document}
	
% 定义所有的eps文件在 figures 子目录下
\graphicspath{{figures/}}

%%% 封面部分
\frontmatter
\ctitle{基于影响域分析的软件补丁兼容性检测}
\cdegree{工程硕士}
\cdepartment[软件]{软件学院}
\cmajor{软件工程}
\cauthor{潘晓梦} 
\csupervisor{贺飞副教授}

\edegree{Master of Software Engineering} 
\emajor{Software Engineering} 
\eauthor{Pan Xiaomeng} 
\esupervisor{Professor He Fei}

% 定义中英文摘要和关键字
\begin{cabstract} 
	众所周知,软件维护是软件开发周期中耗时最长,也是开销最大的过程。软件维护过程中,经常伴随着软件版本不断演进的过程。在实际应用中,补丁程序是最常见的用于软件版本演进的一类程序,它能够实现软件功能增强、漏洞修复等。
	
	补丁程序的局限性在于,它一般都是针对于某个专门版本的代码而设计,通常无法正确的在其他版本代码上使用。然而,在实际开发过程中工程师经常能够遇到这样的问题,如果对其他版本代码重新开发专门的补丁程序,其资源消耗较多。
	
	为此,本文中提出了一套基于语义影响域分析的软件补丁兼容性检测方案,并且实现了相应的工具。语义影响域分析可以挖掘变更对于软件代码的语义影响,根据该分析过程中得到的语义影响域,我们能够进行补丁间的语义冲突检测。通过在Eclipse JDT Core上的实验,我们发现该工具确实能够发现软件补丁的兼容性问题,本文中所提出的解决方案是可用的、可靠的。
\end{cabstract}
\ckeywords {软件维护,语义影响域,补丁,兼容性检测}
\begin{eabstract} 
		As well known,software maintenance is the longest and most cost activity in the software developing cycle.In software maintenance,the software system sometimes got evolved.In pratical application,patch is the most common program which used for software evolving,it can enhance it's functionality,fix it's bug and so on.
		
		The limitation of patch program is that it's usually designed for a specific version of software so that it cannnot be applied to annother version directly.However,it is a common question happended during the software development and developing another patch program for this specific version seems too costly.
		
		Under the above circumstance,we propose a solution called software patch compatibility checking based on semantic impacted area analysis for this problem and develop a corresponding tool.The semantic impacted aread analysis can mines semantic impact introduced by software change from the code.And using it's result we could check if there is a semantic confltct issue between the patches.After experimenting on the Eclipse JDT Core project,the tool is proven to having the ability to find out compatibility issues between the patches and this solution is applicable and sound.  
		
		
%		The software system will continue evolving in its whole lift time,and patch is a common way to accomplish the task to update or fix bug and so on.As the software evolving is rapid, how to identify whether a patch for a specific version of software is applicable for an newer version of the same software.We here present a so called patch compatibility analysis to try to fix this problem.It consists of three kind of analysis—program differencing which is used to output the structural difference of the two versions,change impact analysis which is used to analysis the impact of every change including dependency and impacted expressions/statements and so on.And at last,the conflict analysis used to determine whether the two patches’ impact are confictual.
\end{eabstract}


\ekeywords{software maintenance,semantic impacted area,patch,compatibility checking}
% 设置 PDF 文档的作者、主题等属性
\makeatletter
\thu@setup@pdfinfo
\makeatother
\makecover


% 目录
\tableofcontents


%%% 正文部分
\mainmatter
\chapter{绪论}
\section{研究背景和意义}

软件维护(Software Maintenance)是软件开发周期中耗时最长、开销最大的过程。随着外部环境和用户需求的不断变化,软件系统需要随之进行适应和调整,同时也需要修复在实际运行中暴露出来的问题。

在软件演进的过程中,软件系统可能会由于各种各样的原因而发生更新行为,这是软件维护周期中的常见活动。为了进行软件

补丁(Patch)就是这样一类可以用于完成修补程序漏洞、增强软件功能、改善程序性能等任务的程序。补丁在工业界中得到了广泛的实际应用,是软件维护过程的重要组成部分。

补丁一般是通过Diff工具而产生的文本数据,用于表示以行为单位的程序间差异性。一般而言,Diff工具属于程序差别分析工具中的一种,单纯进行文本的差异性比较,并给出行级别的差异性信息。然而这样产生出来的差异性信息(Patch),只包含了文本差异,而失去了语法、语义等差异信息。

虽然得到了广泛应用,但补丁程序仍然具有一定的局限性,它一般只针对某个专门软件版本而开发,对于软件演进过程中获得的新版本而言,我们无法确定补丁程序是否也能适用于新版本的程序。然而补丁程序的应用一般都具有特定的目的,例如功能升级、漏洞修补等,新版本的程序中很可能仍然需要补丁程序的应用来完善自身。因而在实际的软件维护过程中工程师往往不得不重新去开发针对该新版本的补丁程序,对人力成本和时间资源造成了许多浪费。

究其缘由,主要在于补丁程序通常会引入软件变更(Software Change),使得软件源代码在应用这些变更后可以实现功能修复、版本升级等目的。使用Diff工具所产生的补丁文件一般会以行为单位引入变更。

然而,软件变更不止会局限于对被修改的某行产生作用,事实上,由于软件代码的耦合性,单行变更就足以将变更带来的影响广泛地传播到软件系统的其他部分中去。因此,软件变更不仅会使软件系统发生结构化的变更,还会产生语义上的变化。例如,对赋值语句的修改可能会影响到后续引用到该变量的条件语句。

因此,如何确定补丁程序对于软件其他版本的适用性就成为了一个较复杂的问题,这主要是由于补丁程序所引入的软件变更和版本升级所引入的软件变更可能是互斥的,或者说是部分互斥的。更准确的来讲,即变更所影响到的的程序语法和语义层面之间的互斥性。

可见,要完全确定补丁程序对其他版本代码的兼容性是一件事实上比较困难的事情,目前就我们所知尚无这方面的相关工作出现。而且这方面的工作比较具有实用价值,能够解决工业界的实际问题,因而本文尝试去解决这样一个软件补丁兼容性检测的问题。

\section{本文所要解决的问题与主要工作}

软件系统总是处于不断演进的过程中的。补丁(Patch)是一种常见的软件系统版本演进的方法,常用于修补漏洞和缺陷、改进性能、升级等。在实际应用中,补丁往往针对某一个特定版本的代码而设计,然而现代软件系统往往更新换代较快,因而面临着将补丁应用于其他版本的代码时可能失效的问题。为了解决这个问题,我们为此提出了进行补丁兼容性分析的想法。

为此,我们考虑这样一个问题,针对某软件系统s,假设已有版本$v_1$和$v_2$,其中补丁$p_1$是针对版本$v_1$而设计,应用后能使其演进到版本$v_3$,问$p_1$是否能够应用于版本$v_{2}$,并保证不会发生冲突。

\subsection{应用场景}
\label {app}

本文中的主要问题在于如何判定一个针对给定程序版本的补丁,它是否能够应用于其他版本,并且与其兼容?为了使读者对这一问题更加清晰,我们可以考虑这样一个应用场景,并且提出相应的前提假设:

\begin{itemize}
	\item 某个团队正在对一软件进行开发工作,该软件已推出一个正式版$v_{1}$,现在正在进行$v_{2}$版本的开发。
	
	\item 该团队使用版本控制系统进行版本控制,使用Github作为代码托管和团队协作工具。
	
	\item 该软件存在一个第三方开发的针对$v_{1}$版本的的补丁p,该补丁可以增强该软件的功能。
	
\end{itemize}

我们希望知道,该补丁p是否能够应用于正式版$v_{2}$,并且补丁p中的变更不仅能够被正确地应用,还不影响软件版本$v_{2}$的功能正确性?

\subsection{问题解读}

事实上,对该问题而言,其答案可以分为多个层面来回答。

从语法角度出发,则该问题主要关注的是在应用过程中是否会造成语法结构上的错误,如果能够将补丁$p_1$成功应用于程序版本$v_2$,则认为该补丁是可以兼容于新版本$v_2$的。语法结构上的错误可能是多种多样的,例如补丁中的要修改的代码不在原位置或已被删除、补丁中要添加的代码已经在该文件中存在等等,不一而足。

从语义角度出发,则单纯的语法兼容并不能够完全解答这个问题。某行修改可能会影响多处源代码,从而导致程序的行为发生变化。因而从语义层面而言,我们需要保证补丁$p_1$对程序版本$v_2$造成的语义影响不会波及到从版本$v_1$演进到$v_2$时所造成的语义变化。也就是说我们需要保证补丁$p_1$和$p_2$之间不会存在冲突——即互斥子集。

语法层面的回答很容易就能给出,现有的版本控制系统等都能在一定程度上给出相应的答案。而对于语义层面的答案来说,就目前所知尚未有这方面的工作。

因而本文将着重从语义层面去尝试解决这个问题。更具体的问题描述可以同样参考章节\ref {define_problem}。


\subsection{主要工作}

目前而言,本文的主要工作包括以下几个部分:
\begin{enumerate}
	\item 提出软件补丁兼容性检测问题。
	\item 提出解决该问题的一套解决方案,包括:
	\begin{itemize}
		\item 补丁应用:将转为某个版本代码而设计的补丁应用于其他版本代码。
		\item 语义影响域分析:分析并获取软件变更对代码的影响域。
		\item 冲突分析:根据得到的影响域,分析是否存在语义冲突。
	\end{itemize}
	\item 根据解决方案给出了具体的工具实现,并在工业界的实际项目上进行了实验。
\end{enumerate}

%\begin{enumerate}
%	\item 补丁应用。
%	\item 语义影响域分析。
%	\item 补丁冲突分析。		
%\end{enumerate}

%可以具体定义如下,其中structure意为程序语法结构,可以采用不同级别的程序语法结构(如statement、basic block等)进行分析,来获得不同粒度的影响范围:
%	\begin{enumerate}
%		\item
%		change(s) = { structure | structure属于p,并且structure 发生了变更},该集合可以采用程序间差异分析获得。
%		
%		\item
%		impact(s) = {structure| structure属于p,并且structure 受到change(s)的影响},该集合可以采用变更影响分析获得。
%		
%	\end{enumerate}
%最后得到的impact(s)即为我们所需的变更的语义影响范围。

%所谓的补丁版本迁移工作是指,由于补丁$p_1$本来是适用于程序版本$v_1$的,如果想要适用于程序$v_2$,可能需要进行一定的版本迁移工作。该部分工作可以从语法层面给出兼容性的答案,并为了实现补丁兼容性而进行一定的语法结构修正,主要使用版本控制系统和文本合并工具完成。
%
%所谓的语义域分析工作是指主要采用程序间差异分析、变更影响分析等手段来对界定补丁$p_i$对程序$v_i$所造成的语义影响。在有了补丁$p_1$和$p_2$的语义影响域后,我们对其进行重叠判断,即可找到冲突。该语义影响域分析(Semantic Impacted Area Analysis)过程可以参考章节\ref {sia}中的详细叙述。
%
%而在有了语义影响范围之后,就可以进行冲突分析的工作。通过影响域的重叠判定,我们可以获得两个补丁间的冲突,通过人工分析的辅助,我们可以判定该互斥子集是否存在误报(false positive),并筛选出真正的互斥子集。
%
%我们可以简单的认定如果两个补丁间如果存在真互斥子集,那么他们之间就存在冲突。而实际上,真互斥子集中的一些互斥变更,根据我们的定义他们是互斥的,但在实际应用中他们其实仍然可能是互相兼容的。
%
%如果对真互斥子集进行人工的冲突分析,可以将这些情况进一步剔除。
%
%可见,我们给出的冲突定义是比实际冲突更加严格的定义,可能会造成一定的过高估计(over-estimate)。在第\ref {exp}章的实验中我们会给出具体的介绍。


\section{本文组织结构}

本文主要包括六个章节,第一章是绪论,介绍本文的研究背景和主要工作;第二章主要介绍与本文所述内容相关的国内外的工作;第三章主要介绍了补丁兼容性分析的具体方法;第四章主要介绍了如何将各阶段的不同分析方法进行整合,形成具体的流程;第五章介绍了实验过程和结果;第六章主要对本文的工作进行了总结,并提出了进一步的工作方向。

%\section{本章小结}
%本章中主要提出了如何进行软件补丁兼容性检测的问题,并介绍了其实际意义。其次介绍了本文中为了解决该问题而做出的主要工作,最后给出了本文的组织结构。


\chapter{相关工作}

本章中主要介绍与本文相关的国内外工作,主要包括对变更影响域分析的子过程和相关工具的调研。
\section{程序间差异性分析}
\label{relate_diff}

对于软件演进分析而言,如何确定一个程序的不同版本之间的变更是一个关键性的问题\cite{kim2013identifying}。程序间差异性分析能够通过分析同一程序的不同版本间的差异,来确定版本间的变更集合\cite{lahiri2010differential,winstead2003towards}。

按分析的深度而言,程序间差异性分析可以分为三类:
\begin{itemize}
	\item 文本差异:单纯对比文本间的不同,这是最简单也最广泛应用的分析方法,如Unix Diff工具。
	\item 语法差异:对比并获得源代码间语法结构上的不同。
	\item 语义差异:对比并获得源代码间语义层面上的不同。
\end{itemize}

现有的帮助工程师进行软件维护和演进过程的工具往往都受限于低质量的变更信息。例如,源代码的变更信息往往都存储于版本管理系统中(如CVS)等,它们会追踪对某个特定文件的文本行的增加/删除等操作,但没有考虑代码中的结构化变更。

考虑到源代码能够以抽象语法树(Abstract Syntax Tree)的形式进行表达,可以采用树间差异分析的方法来抽取出这些变更信息。Change Distilling就是这样一类进行树间差异分析的算法\cite{fluri2007change,gall2009change}。该算法能够从两棵AST之间寻找匹配节点,并找到一个能够令一棵树转化为另一棵树的最小变更集合,该变更集合即为所求的程序间差异。而且由于是从AST中抽取信息,该算法可以获取语法结构上的变更信息。

Change Distilling中采用二元字符串相似性来匹配源代码语句,并使用子树相似性来匹配源代码结构(如语句、循环等)。在寻找变更集合时,它采用基本的树变更操作来描述源代码的变更,包括更新、删除、增加等。在实际的使用过程中,该算法可能会受限于如何找到数量合适的移动操作。

\section{程序变更影响分析}
\label{relate_impact}
软件维护是软件开发周期中最为复杂、成本最高的劳动密集型活动。软件产品需要跟随用户需求的变更而进行适应和变化,而软件变更可以帮助软件实现这种维护过程。事实上,软件变更是软件维护过程中的基础组件,它可能来自于新的需求、缺陷修复、变更请求等,然而将变更应用于软件时,它们会不可避免的带来一些副作用,导致可能会与原软件的其他部分发生冲突。

而变更影响分析(Change Impact Analysis)正是这样一类用于确定变更对于软件其他部分影响的技术集合\cite{li2013survey},它在软件开发、维护和测试等过程中都起到重要的作用\cite{acharya2011practical}。一般而言,变更影响分析可以用于程序理解、变更影响预测、影响追踪、变更传播、测试用例的选取等过程。

变更影响分析方法可以分类如下:
\begin{itemize}
	\item 基于可追踪性的变更影响分析\cite{de2008traceability},它追踪两个不同抽象级别的软件元素之间的依赖性,其目的在于链接不同类型的软件工件(如需求、设计等)。

	\item 基于依赖的变更影响分析\cite{law2003incremental},它致力于衡量变更的潜在影响,并试图分析程序语法结构间的关系,即程序实体间的语义依赖。这类变更影响分析主要在源代码级别进行研究。
\end{itemize}

软件变更可能导致意料之外的副作用,而变更影响分析的目的就在于找到这些副作用(Side Effect或者Ripple Effect)\cite{bohner1996software},并防止之。
变更影响分析从分析变更请求和源代码开始,最后能够得到估计影响集合(Estimated Impact Set),与真实影响集合(Actual Impact Set)相比该结果可能存在一定的误差。

整个软件变更影响分析的过程可以大致划分为如下流程\cite{de2008traceability,bohner2002software},更详细的流程可以参考图\ref {变更影响分析process}。

%\begin{enumerate}
%	\item 
	该分析过程需要输入变更集合,然后对变更请求进行分析。该步骤即特征定位(feature location),用于找到源代码中相应功能的起始实现位置\cite{biggerstaff1993concept}。该分析过程将衡量变更集合引入的影响。该步骤是目前大部分变更影响分析技术的重点,其主要的两类包括:
%	\item 
	
	\begin{itemize}
	
	\item 静态分析:包括历史分析、文本分析、结构分析等\cite{sun2012comparative,kagdi2007survey}。静态分析主要分析程序的语法、语义或者历史依赖,容易产生许多误报(False Positive)。
		\begin{itemize}
		\item 结构分析着重于分析程序间的结构依赖性并构建依赖关系图
		\item 文本分析根据程序中的注释和标识符提取出其概念依赖性
		\item 历史分析能够从多个软件版本的演进过程中挖掘相关信息
		\end{itemize}
%		\begin{figure}[H]
%			\centering
%			\includegraphics{chap02_02}
%			\caption {静态分析过程}
%		\end{figure}
	
	\item 动态分析:包括在线分析和离线分析。该分析过程需要给出特定输入,并依赖程序运行时所收集到的信息来进行分析(如运行时的路径追踪和覆盖信息等)\cite{law2003whole}。该分析过程所得到的影响集合往往比静态分析的精度更高,但其开销也相应更大,且容易错报(False Negative)。
%		\begin{itemize}
%		\item 在线变更影响分析使用程序运行时收集的信息来进行
%		\end{itemize}
		
%		\begin{figure}[H]
%			\centering
%			\includegraphics{chap02_03}
%			\caption {动态分析过程}
%		\end{figure}	

		
	\end{itemize}
%\end{enumerate}

\begin{figure}[H]
	\centering
	\includegraphics[height=.6\columnwidth]{chap02_01.jpg}
	\caption {变更影响分析过程}
	\label {变更影响分析process} 
\end{figure}

近年来,学术界中实现了某些以变更影响分析技术为支撑的工具,这些工具通常会利用变更影响分析得到的影响集合来完成后续的分析过程,帮助软件进行维护和演进。下面给出相关的简要介绍:
\begin{itemize}
	\item Chianti:支持Java语言,可作为Eclipse插件使用\cite{ren2004chianti}。该工具首先使用回归测试来分析变更前的程序是否能正常使用,若回归测试失败,则利用Chianti工具实施变更影响分析,该分析过程通过将软件变更拆分成若干原子变更来分析变更之间的依赖关系。最后结合原程序生成某种中间表示形式并找到可能影响到测试用例运行的代码位置。
	
	\item JRipples:支持Java语言,可作为Eclipse插件使用\cite{buckner2005jripples,rajlich2004incremental}。该工具利用依赖关系图自动标注可能被变更的类所影响的其他类,并提示用户其变更的可能影响传播路径。该工具的分析结果可进行人工修正。
	
	\item ROSE:支持用CVS工具进行版本管理的Java项目,可作为Eclipse插件使用\cite{zimmermann2005mining}。该工具需要挖掘软件代码仓库,当用户对代码进行变更时,提示用户某些其他变更可能与之相关(其形式类似于“变更了该函数的人通常还变更了另一个函数”)。
	
	\item jpf-regression:支持Java语言,可用作Eclipse插件或直接作为命令行工具使用\cite{person2011directed}。该工具利用程序切片技术进行变更影响分析,使用得到的影响集合来驱动符号化执行,找到可能被变更影响到的程序行为。
\end{itemize}



\section{相关工具}
\label{relate_tool}
	本节主要介绍实验采用的相关工具。
	

	\subsection{git}		

git是一个分布式的版本控制系统,最初由 Linus Torvalds在2005年为Linux内核而开发,现在已经成为最流行的版本控制系统。

与CVS和SVN等集中式的C/S版本控制系统不同,git是一种分布式的版本管理系统,每个本地git工作目录都具有完整的历史数据和版本追踪能力,无需网络连接或服务器端的支持。
      
本文主要采用git作为版本管理与合并的工具。

	\subsection{Beyond Compare}
		
Beyond Compare 是一款内容比较工具,可用于文件、目录、压缩包等数据间的比较,横跨Windows、Mac OS X和Linux三大操作系统,可用作常见版本控制系统的第三方文本比较与合并工具。
      
本文主要采用该工具作为文本比较和合并工具,用于解决补丁版本迁移时可能遇到的冲突。


	\subsection{jpf-regression} 
变更影响分析常用于衡量软件变更的潜在影响,其分析结果通常可用做其他程序分析技术的输入,例如回归测试可利用变更影响分析来确定程序的哪些部分需要进行再分析。由于单行变更就足以引发广泛的未知影响,变更影响分析在软件的演进和维护过程中扮演着重要的角色。\cite{rungta2012change}

目前,大部分的自动分析工具都以程序语法结构的形式描述变更的影响,如函数和语句等。基于依赖的分析方法一般通过分析程序组件间的内部关系来衡量变更的影响,这类技术通常使用程序位置信息来描述变更的影响,缺失了受影响代码位置的相关运行路径信息。这类信息往往对程序行为的验证、调试等工作帮助很大,而且能够将需要关注的代码范围缩小,使得只需关注受影响的程序行为集合即可。

(Directed Incremental Symbolic Execution )DiSE\cite{person2011directed,yang2014directed}方法能够结合静态分析的效率和符号化执行的精度等优点,该方法能分析限制在方法内部的变更影响,并描述程序变更对其行为的影响。

jpf-regression是DiSE方法的工具实现,它基于Java Path Finder框架\cite{havelund2000model}实现,支持Java语言,可作为Eclipse的插件使用。本文将采用该工具来实现变更语义影响分析过程。下面对该方法中的影响分析算法进行简介。

DiSE方法采用程序切片技术来衡量变更对代码中其他部分的影响,其生成的影响集合可以用于引导符号化执行来分析受变更影响的程序行为,并生成相应的路径条件(Path Condition),这些路径条件描述了受影响的程序行为,在分析结束后可以利用SMT等技术进行路径求解,用于后续的验证、调试等过程。

在DiSE方法中,程序变更影响分析是其后续分析过程的基础,其变更影响分析技术的主要特点包括:
\begin{enumerate}
	\item 粒度:基本块。即变更影响分析过程中,以基本块为单位进行影响集合的计算。
	\item 影响范围:方法内部。即将单次影响集合计算的范围局限于方法内部,最后得到变更对其所属方法内部的其他语句所造成的影响。
	\item 影响来源:主要从控制流和数据流两个方面考虑变更所造成的影响,采用语句间的控制依赖和数据依赖关系进行影响计算。
\end{enumerate}

DiSE方法中的影响集合主要分为两类:
\begin{itemize}
	\item ACN:受影响的条件语句节点(Affected Conditional Nodes)。
	\item AWN:受影响的赋值语句节点(Affected Write Nodes)。
\end{itemize}

为了得到这两类影响集合,DiSE中使用四条规则进行迭代计算。从原始的变更集合出发,不断应用规则向外扩展,最后得到的闭包即为所求的影响集合:
\begin{enumerate}
	\item 如果ACN中有一个节点$n_i$,且存在一个条件节点$n_j$,其中$n_j$控制依赖于$n_i$,那么将$n_j$加入到ACN中。
	\item 如果nj是一条赋值语句,且控制依赖于ACN中的节点$n_i$,那么$n_j$加入到AWN中。
	     
%	前两条公式表明,如果条件语句和赋值语句控制依赖于变更后的CFG中的节点,那么他们就应当被加进来。
%	     
	\item 如果AWN中的一条赋值语句节点$n_i$对于变量的赋值在条件语句$n_j$中被使用了,且CFG中存在一条从$n_i$到$n_j$的路径,那么将$n_j$加入到ACN中。

	\item 如果一个写语句节点$n_i$对于对于变量的赋值在AWN或ACN中的某个节点$n_j$被使用了,且CFG中存在一条从$n_i$到$n_j$的路径,那么将$n_i$加入到AWN中。

\end{enumerate}



\subsection{ASTro}	

	本文采用的程序间语法差异性分析工具是由内布拉斯加大学林肯分校的Josh Reed,Suzette Person和Sebastian Elbaum等人所开发的ASTro,它是jpf-regression工具自带的前置工具,用于比对两个不同版本的源代码并获取其抽象语法树上的差异,并将得到的变更集合以XML格式输出。该工具支持Java语言。
	
\section{本章小结}
本章主要介绍了相关的国内外工作。
章节\ref{relate_diff}介绍了程序间差异性分析的相关工作。
章节\ref{relate_impact}介绍了变更影响分析的相关工作。
章节\ref{relate_tool}介绍了本文中所用到的相关工具。
	
\chapter{软件补丁兼容性检测方法}

\section{兼容性问题}
\label {sec_problem}

回顾~\ref{sec:background} 中提到的应用场景:
\begin{itemize}
	\item 某项目团队在开发过程中使用了某开源第三方软件,并针对该开源软件开发了专门的补丁以适用于本项目。
	\item 当第三方软件更新到新版本,在集成该新版本的时候,想知道原补丁是否还能继续使用。
\end{itemize}

考虑到版本更新也可以使用补丁来完成,该问题就可以从另一种角度来看待。在集成该新版本的时候,原补丁是否能够继续使用的问题就可以转化为原补丁是否会和升级补丁产生冲突的问题。也就是说,软件补丁对于其他版本的适用性可以从补丁之间的冲突这一角度来考虑。

由于每个补丁都可以视作一系列的变更集合,其中的每条变更都会修改原有代码版本的语法结构,并可能会对其他语法结构造成语义上的影响,那么从这一角度来看时,补丁兼容性问题就得到了简化。我们可以找到每个补丁中的变更所影响到的程序语法结构的集合,并探讨这两个集合之间是否存在一定的交集,显然,如果两个补丁中的变更影响到了相同的语法结构,该位置上的语法结构受到了双重影响,该位置可能出现冲突。例如对某条件判断语句而言,原补丁在某处对其引用的变量值进行了修改,使得条件语句中该值增加,而升级补丁中在另外一处对该变量的修改则会导致条件语句中其值减少,显然,这样的双重影响是矛盾的。

当然,在某些情况下,这样的双重影响也是可以共存的。例如上文条件判断语句的例子中,如果两个补丁中的修改都会导致该条件语句中引用的某变量值增加,那么这样的双重影响就可能是兼容的。

这类双重影响并没有严格的规则来判断是否一定冲突或不冲突,因此,只能通过人工分析来辅助判断。

综上所述,软件补丁的兼容性检测问题可以归结为多次变更之间的冲突检测。

\section{检测方法概述}
\label {sec_method}

本文中提出了一种软件补丁兼容性的检测方法,该方法的整体流程可以参考图~\ref{all_flow}。该方法分为三步。首先,将软件补丁应用于其他版本的代码上,并防止该过程中引入语法错误。其次,我们需要找到补丁中的变更所影响到的其他语法结构,也就是后文中所提出的变更影响域的概念。最后,我们根据找到的变更影响域,进行冲突检测。该方法的输入包括$v_1$版本和$v_2$版本的代码,以及适用于$v_1$版本的补丁p。其中版本$v_1$为旧版本,版本$v_2$为新版本。

\begin{figure}[H]
	\centering
	\includegraphics[height=.6\columnwidth]{chap03_all_now}
	\caption {解决方案}
	\label {all_flow}	
\end{figure}

该方法可以检测版本$v_1$到版本$v_2$的升级过程中所引入的变更是否会与补丁p在版本$v_2$中所引入的变更发生冲突。之所以选择以版本$v_2$作为基准来进行冲突检测,是由于在实际过程中,应用该检测方法之前还需要将补丁p应用至代码版本$v_2$,以便完成补丁应用的过程。

本文采用了版本合并的方法来解决该过程中可能引入的语法错误,其流程简述如下:
\begin{enumerate}
	\item 将补丁p应用到版本$v_1$,获得版本$v_3$。
	\item 采用三路归并算法将版本$v_2$和$v_3$进行合并,获得新版本应用补丁后的代码,其版本为$v_4$。
	\item 解决分支合并中可能出现的冲突问题。
\end{enumerate}

最后所得到的$v_4$版本代码即为我们所需要的在版本$v_2$上应用了补丁p中变更的新版本代码。

由于该合并过程较为简单,可以直接使用git等版本控制系统完成,以后的章节中将不再赘述,只在实验部分给出该过程相应的结果与分析。本文在以后的章节中将直接考虑对从版本$v_1$到$v_2$的升级过程中引入的变更与从版本$v_2$到$v_4$的升级过程中引入的变更进行冲突检测。

下面对该检测方法中所涉及到的过程进行概述。

首先,该检测方法中提出了一种软件变更影响域分析方法,该分析方法能够找到不同版本的代码间的变更集合,并以此为依据进行分析,最后找到变更集合所对应的语义影响域,即变更影响域,该变更影响域中包含了所有受到变更集合影响的程序语法结构。具体的变更影响域分析过程和变更影响域等相关概念的定义可以参考章节~\ref{chap_impact} 中的叙述。

其次,该检测方法根据得到的不同变更影响域,使用软件变更冲突检测方法判定这些变更影响域之间是否存在着语义上的冲突。该软件变更冲突检测方法的相关实现和定义可以参考章节~\ref{chap_conflict} 中的叙述。

本文根据此处提出的兼容性检测方法进行了相应的工具实现。按照兼容性检测方法的流程,该检测工具可以划分为两个模块:
\begin{itemize}
	\item 影响域分析模块:实现解决方案中的软件变更影响域分析过程。
	\item 冲突判定模块:实现解决方案中的软件变更冲突检测方法。
\end{itemize}

这些模块的实现过程可以分别参考相关章节中的叙述。

\section{本章小结}
本章概括介绍了软件补丁兼容性问题和其检测方法。
章节~\ref{sec_problem} 中介绍了补丁兼容性问题。
章节~\ref{sec_method} 中对补丁兼容性检测方法和其工具实现进行了简要介绍。详细的介绍参见后续章节。



%\chapter{软件补丁兼容性检测方法}
\section{兼容性问题}
\label {define_problem}
第一章中对本文所要解决的问题进行了简要介绍。下面对该问题进行正式的定义。

%\begin{problem}
%	\label {origin}
	针对某软件系统s,假设已有版本$v_1$和$v_2$,其中补丁$p_1$是针对版本$v_1$而设计,应用后能使其演进到版本$v_3$,问$p_1$是否能够应用于版本$v_{2}$,并且应用之后是否会发生冲突?
%\end{problem}

考虑到补丁间不兼容的实质是因为在应用时和应用后出现了错误,可以具体将错误分为两类:

\begin{definition}
	兼容性错误——将补丁$p_1$和补丁$p_2$先后应用于同一版本代码$v$时和之后所出现的程序错误。主要分为两类:
	\begin{itemize}
		\item 语法错误:即在应用一补丁$p_1$之后,再次应用另一补丁$p_2$时出现语法结构上的错误。
		\item 语义错误:将补丁$p_1$中的变更和补丁$p_2$中的变更应用于源代码之后,由于语义上的变更互相冲突而导致出现了功能上的错误
	\end{itemize}
\end{definition}

其中,语法错误易由编译器发现,而语义错误深藏于代码中,较难发现。可见,该问题的核心包括两点,即:
\begin{itemize}
	\item 如何将一个专为版本$v_1$设计的补丁$p_1$成功应用于版本$v_2$上。即如何消去补丁应用过程中的语法错误。
	\item 如何检测补丁在应用于新版本之后是否会与该版本代码产生冲突?即如何检测补丁应用之后的语义错误。
\end{itemize}

下面分别对这两点进行分析和阐述。

\subsection{补丁应用}

实际上,补丁程序是专门为某一版本的软件系统所设计,因而往往不能直接应用到其他版本的代码上,而且强行应用往往会出现很多错误。常见的问题可能包括:

\begin{enumerate}
	\item 补丁程序中所提及的行不在原位置。
	\item 补丁程序中所要删除的行其实已被删除,或者所要添加的行其实已被添加。
	\item 补丁程序中所提及的文件未找到。
\end{enumerate}

由于有这些可能的问题存在,导致我们无法直接将补丁程序应用到其他版本的补丁上。因此,我们考虑如何去解决问题的第一点:

由于补丁$p_1$针对版本$v_{1}$设计,当尝试将其应用到版本$v_{2}$上时,可能会出现程序语法结构上的错误,考虑怎样将$p_1$成功应用到版本$v_{2}$上,使得该过程不会引入程序语法错误。

%\begin{problem}
%	\label {patch_reversion}
	
%\end{problem}

	假若用$Code$表示源代码,$Patch$表示补丁,即变更的集合。用$patch(p,v)$函数表示将补丁$p$应用于版本$v$的过程。用$Pat$函数判断某个补丁是否适用于某个版本的代码。用$error$函数表示对源代码的语法检查,其结果表明源代码中是否存在语法错误。其中$v_k$表示代码的第k个版本,$p_i$表示第i个补丁。
	
	那么我们可以将其定义如下:
\begin{definition}
	给定$p_i$,$v_k$,问$p_i \not \subset Pat(v_k)$时,能否使得$error(patch(p_i,v_k)) == False$?其中$p_i \subset Patch,v_k \subset Code,i,k \subset \mathbb{N}$。
\end{definition}
	
	

%\begin{definition}
%	$patch: Patch \times Code \mapsto Code$。$patch(p_i,v_k) = v_m,v_k,v_m \subset Code,i,k,m \subset \mathbb{N}$。
%\end{definition}

%\begin{definition}
%	$Pat: Code \mapsto \{Patch\}$。用于表示代码到其对应的补丁集合的映射,可用于判断某个补丁是否适用于某个版本的代码。
%\end{definition}

%\begin{definition}
%	$error: Code \mapsto Boolean$。error函数用于对源代码的语法检查,其结果表明源代码中是否存在语法错误。
%\end{definition}

%\begin{definition}
%	\label {app_formal}

%\end{definition}

可见,要解决问题的第一点,其核心在于如何给出合适的$patch(p_i,v_k)$函数,使得$p_i$应用于版本$v_k$之后不会出现语法错误。

考虑到$patch(p_1,v_1) = v_3$,为了将$p_1$应用到版本$v_2$上获得版本$v_4$,我们可以采用某种归并算法,将版本$v_3$和版本$v_2$进行合并即可。合并后我们就等于同时拥有了$p_1 = diff(v_1,v_3)$和$p_2 = diff(v_1,v_2)$这两个补丁中的变更。

整个过程可以形式化定义如下。实际上也就是说我们用$merge$函数实现了将补丁应用于其他版本的代码的功能。

\begin{definition}
	$merge: Code \times Code \mapsto Code$。$v_s = patch(p_i,v_l) = merge(v_m,v_l)$,使得$\forall c_a \subset p_i,\forall c_b \subset p_j,c_a,c_b \subset p_k$,即$p_k = p_i \cup p_j$。其中$v_m = patch(p_i,v_n),p_j = diff(v_n,v_l),p_k = diff(v_l,v_s),p_i \not \subset Pat(v_l),p_i,p_j \subset Pat(v_n),p_k \subset Pat(v_l),v_l,v_m,v_n,v_s \subset Code,a,b,i,j,k,l,m,n,s \subset \mathbb{N}$。
\end{definition}

\subsection{补丁冲突}

就问题的第二点而言,直接去寻找冲突的存在是一件让人困惑的事情,什么是冲突?为什么会发生冲突?怎么进行检测?

因此本文考虑从另一个角度来看待这一点。

实际上,由于从版本$v_1$到版本$v_2$的演进过程同样可以使用补丁来完成。那么可以考虑将这个演进过程表述成为使用补丁进行变更的过程,以得到另外一个用于进行版本更新的补丁。

\begin{definition}
	\label {define_diff}
	$diff : Code \times Code \mapsto Patch$。其中$p = diff(v_i,v_j) \iff patch(p,v_i) = v_j,i,j \subset \mathbb{N},p \subset Pat(v_i)$。该函数用于求解两个不同版本的程序$v_i$和$v_j$之间的差异性,其结果即为补丁p。
\end{definition}

从问题中可以发现,$p_{1} = diff(v_{1},v_{3})$。如果同样给定$p_{2} = diff(v_{1},v_{2})$,并且问题的第一点已经得到了妥善的解决,那么,对于问题的第二点而言,如何检测补丁在应用于新版本之后是否会与该版本代码产生冲突就可以规约如下:

%\begin{problem}
%	\label {compatible}
	给定针对某同一版本代码$v$的补丁$p_1$和$p_2$,问分别应用于$v$之后,这两个补丁对于该版本代码的语义影响是否会产生冲突?
%\end{problem}

这样一来,问题的实质就清楚了。显然,该问题是由于两个补丁同时对同一版本代码造成了语义影响而引起的。那么为了回答这个问题,我们需要明确的知道到底什么是补丁对代码的语义影响,以及什么是冲突。

本文首先讨论什么是补丁对代码的语义影响。



%\begin{definition}
%	
%\end{definition}

事实上,如果给出不同的依赖关系的定义,那么就会得到不同类型的影响。可见,这里所谓的语义影响即是程序代码间的耦合关系。常见的依赖关系包括控制依赖和数据依赖等。

下面给出语义影响的定义。其中Structure用于表示程序代码结构,Sturct函数为从源代码到其所有代码结构集合的映射。其中依赖关系可以包括控制依赖、数据依赖等。

%\begin{definition}
%	$Structure \to Class \mid Method \mid Statement \mid ...$。Structure用于表示程序代码结构,可以任意替换为对应的子类型。
%\end{definition}

%\begin{definition}
%	$Struct: Code \mapsto \{ Structure \}$。
%\end{definition}

\begin{definition}
	$im: Structure \mapsto Structure$。影响关系定义为一个从代码结构到代码结构的映射关系,$im(structure_i) = structure_j$用于表示代码结构$structure_i$受到$structure_j$影响,$i,j \subset \mathbb{N}$。
\end{definition}

\begin{definition}
	$depend: Structure \mapsto Structure$。依赖关系定义为一个从代码结构到代码结构的映射关系,$depend(structure_i) = structure_j$用于表示代码结构$structure_i$依赖于$structure_j$,$i,j \subset \mathbb{N}$。
\end{definition}

\begin{definition}
	$\forall structure_i,structure_j \subset Struct(v_k),  im(structure_i) = structure_j \iff depend(structure_i) = structure_j$,其中$v_k \subset Code,i,j,k \subset \mathbb{N}$。
\end{definition}

从中可见代码结构$structure_i$受到$structure_j$的影响,当且仅当代码结构$structure_i$依赖于$structure_j$。


%\begin{definition}
%	\label {define_conflict}
%
%\end{definition}

因此,可以认为,语义冲突(Confilict)是指两个补丁所对应的不同变更集合之间存在某两条互斥的变更,其中两条变更互斥指二者的语义互相影响。

若$change$函数用于表示代码结构和其所对应的变更的映射,那么可以将其定义如下。

%\begin{definition}
%	
%\end{definition}

%\begin{definition}
%	$change: Structure \mapsto Change$。用于表示代码结构和其所对应的变更的映射。其中$Change$——指变更。$Patch = \{Change\}$。
%\end{definition}


\begin{definition}
	$conflict: Patch \times Patch \mapsto Boolean$。如果$\exists structure_i,structure_j,structure_k \subset v_t$,使得$im(structure_i) = structure_k \land im(structure_j) = structure_k$,那么$conflict(p_m,p_n) = True$。其中$change(structure_i) \subset p_m,change(structure_j) \subset p_n,p_m,p_n \subset Pat(v_t),v_t \subset Code,i,j,k,m,n,t \subset \mathbb{N}$。
\end{definition}

该$confilct$函数用于判断两个补丁是否发生语义冲突。该函数表示,假如补丁$p_m$中的某条变更修改了代码结构$structure_j$,补丁$p_n$中的某条变更修改了代码结构$structure_i$,并且代码结构$structure_i$和$structure_j$影响到了同一个代码结构$structure_k$,那么就说明补丁$p_n$和$p_m$之间存在着语义冲突。

在有了语义影响和冲突的定义之后,问题的第二点就可以形式化的描述如下:

%\begin{problem}
%	\label {compatible_formal}
	给定$v_k \subset Code, p_i,p_j \subset Pat(v_k),i,j,k \subset \mathbb{N}$,问是否有$conflict(p_i,p_j) == True$?
%\end{problem}

%因此,结合前文中对问题第一点的形式化描述,该问题就可以形式化的定义如下:
%
%%\begin{problem}
%%	\label {origin_formal}
%	给定$p_i,v_k,v_m$,问当$p_i \not \subset Pat(v_k)$时,如果能够使得$compile(patch(p_i,v_k)) == True$,并且$\exists p_j = diff(v_k,v_m)$,那么是否有$conflict(p_i,p_j) == True$?其中$p_i \subset Patch,p_j \subset Pat(v_k),v_k,v_m \subset Code,i,j,k,m \subset \mathbb{N}$。
%%\end{problem}

%\section{问题分析}
%\label {problem_analysis}
%
%\subsection{补丁应用}
%\label {patch_app}

%如前所述,为了解决问题的第一点,主要需要提供合适的$patch(p_i,v_k)$函数。

%实际上可以采用版本控制系统的版本合并功能来作为该函数的具体实现。
%
%版本控制系统中经常需要将其他分支中的版本合并到当前分支中,为了解决不同版本之间可能存在的语法冲突,通常会采用三路归并算法(3-way merge)来对两个不同版本的代码进行合并,合并时以二者共同的祖先版本作为依据进行。



%这实际上也是现在软件开发过程中的常见手段。例如使用git作为版本控制系统时,为了修复某个功能性的bug,我们可以新建一个分支FixBug,然后切换到该分支进行漏洞修复,等到修补完毕后,再将该分支合并到主分支Master中即可。
%
%该过程如图\ref{git_branch}所示。
%
%\begin{figure}[H]	
%	\centering
%	\includegraphics[height=.6\columnwidth]{chap03_git_branch}
%	\caption {git分支切换与合并}	
%	\label {git_branch}
%\end{figure}


%\subsection{补丁冲突}
%\label {conflict}

为了解决问题的第二点,我们主要需要从代码中挖掘中两类信息:
\begin{enumerate}
	\item 代码的变更集合,即两个版本之间代码的差异性。用于进行问题规约。
	\item 变更的影响集合,即变更集合对代码中其他哪些部分会造成影响。用于确定变更造成的语义影响。
\end{enumerate}

在有了这两类集合之后,我们就可以界定出软件变更在语义层面上对代码的语义影响域,我们可以将这整个过程称为语义影响域分析(Semantic Impacted Area Analysis)。

下面首先给出语义影响域的概念。语义影响域(Sematic Impated Area)是指程序的某个限定范围中,直接或间接受到变更影响的程序结构集合。其中程序结构——指程序的基础语法结构,包括类、方法、基本块、语句等不同级别。

%\begin{definition}
%	
%\end{definition}
%
%\begin{definition}
%	
%\end{definition}

可见,语义影响域描述了变更对于程序中其他部分的语义影响。下面给出影响域的定义。

\begin{definition}
	$impact: Patch \times Code \mapsto \{Structure\}$。$impact(p_m, v_k) = \{ \forall structure_j \subset Struct(v_k) \mid structure_j = im(sturcture_i) \land (change(structure_i) \subset p_m \lor structure_i \subset impact(p_m,v_k)),structure_i,structure_j \subset Struct(v_k),v_k \subset Code,p_m \subset Pat(v_k), i,j,k,m \subset \mathbb{N} \}$。
\end{definition}

该函数用于描述补丁$p_m$对于其所应用的某个版本的代码$v_k$的影响域。其中可以将影响关系$im$限定在不同代码结构类型的范围内。

为了挖掘出这两类信息,可以将语义影响域分析划分为两个子过程:
\begin{enumerate}
	\item 程序间差异性分析:获取两个软件版本间的差异性,获得结构化的软件变更信息。该分析过程即为$diff$函数的具体实现。
	\item 变更影响分析:分析软件变更对软件其他部分是否存在影响,并找到受影响的集合。该分析过程即为$impact$函数的具体实现。
\end{enumerate}



在有了软件变更的语义影响域之后,我们可以通过简单地比较两个补丁对代码的影响域是否存在重叠来判定其是否发生了语义冲突。语义冲突(Semantic Confilict)的实质就是指两个补丁所对应的不同变更集合之间存在某两条互斥的变更。其中两条变更互斥指二者对同一行代码进行了修改,或者他们所修改了不同行代码,但其影响传播到了某行相同的代码。

%\begin{definition}
%	\label {define_conflict}
%	
%\end{definition}

%为了更清楚的理解,可以给出如下的形式化定义。
%也就是说:
%\begin{definition}
%	\label {problem}

%	如果$\exists c_m \subset p_i, \exists c_n \subset p_j$,其中$c_m = change(structure_m),c_n = change(structure_n)$, $structure_m \subset Struct(v_k), structure_n \subset Struct(v_k), v_k \subset Code$,并且$i,j,k,m,n \subset \mathbb{N}$。
%	那么有$conflict(p_i,p_j) == true \iff (structure_m == structure_n) \lor (impact(p_i,v_k) \cap impact(p_j,v_k) \neq \varnothing)$,其中$p_i,p_j \subset Pat(v_k)$。
%\end{definition}

可见,如果两个补丁对代码的影响域不存在重叠,那么他们两者之间自然是兼容的,因为他们不仅本身的变更互不干涉,并且他们所影响到的程序结构也互不干涉。在这样的情况下,补丁是可以成功应用到其他版本上的代码,并且可以完成补丁本身的目标的。

如果影响域发生了重叠,那么我们就可以认为补丁之间是可能不相容的,因为补丁所作的变更会对相同的程序结构产生影响。而该部分代码是否发生冲突了还需要人工的判定,因为在不同的场景下,他们可能是兼容的,也可能是不兼容的。而我们无法从代码中获取到足够的信息来进行这样的判定,需要外界对于变更的期望信息作为辅助。

%我们可以将这个过程定义为如下类型的函数,并称之为冲突分析。可见,我们实际上是采用了$isCompatible(s_i,s_j)$函数来作为$conflict(p_i,p_j)$的具体实现。
%
%\begin{definition}
%	$isCompatible(s_i,s_j) : 、\{Structure\} \times \{Structure\} \mapsto Boolean$。其中$conflict(p_i,p_j) = isCompatible(impact(p_i,v_i),impact(p_j,v_j)),i,j \subset \mathbb{N}$。
%\end{definition}

\subsection{总结}

根据前两节中对问题各点的分析,可以得到对该问题的形式化描述:

给定$p_i,v_k,v_m$,问当$p_i \not \subset Pat(v_k)$时,如果能够使得$error(patch(p_i,v_k)) == False$,并且$\exists p_j = diff(v_k,v_m)$,那么是否有$conflict(p_i,p_j) == True$?其中$p_i \subset Patch,p_j \subset Pat(v_k),v_k,v_m \subset Code,i,j,k,m \subset \mathbb{N}$。


可以发现,对于整个问题而言,其解决方法的核心在于:
\begin{itemize}
	\item 如何将补丁成功应用于其他版本?即$merge$函数的具体实现。
	\item 如何检测补丁间的冲突?该部分的解决过程可以分为两步:
	\begin{enumerate}
		\item 如何规约该问题并计算影响域?即$diff$和即$impact$函数的具体实现。也就是整个语义影响域分析的过程。
		\item 如何判定发生了冲突?即$conflict$函数的实现。也就是整个冲突分析的过程。		
	\end{enumerate}
\end{itemize}

因此,整个补丁兼容性问题实际上可以归结为多次软件变更的影响域冲突检测问题。


\section{检测方法}
\label {problem_solve}

考虑前一节中对问题的定义与分析,我们可以给出一个通用的解决方案来解决整个补丁兼容性问题。

\subsection{整体架构}
\label {problem_all}

根据前文中的问题分析,实际上整个兼容性问题的解决过程可以参考图\ref {all_flow}。其输入输出的描述参考表\ref {all_io}。

可见,整个解决方案的实现过程包括三步:
\begin{itemize}
	\item 补丁应用过程。将专为某版本代码而设计的补丁应用于其他版本的代码,即$merge$函数的实现过程。
	\item 软件变更影响域分析过程。分析补丁中的变更所造成的语义影响,即$diff$函数和$impact$函数的实现过程。
	\item 软件变更冲突检测过程。检测补丁间的变更其影响域是否会发生冲突,即$conflict$函数的实现过程。
\end{itemize}

\begin{figure}[H]
	\centering
	\includegraphics[height=.6\columnwidth]{chap03_all_flow_2}
	\caption {解决方案}
	\label {all_flow}	
\end{figure}

\begin{table}[H]
	\caption{输入输出对照表}
	\label{all_io}
	\centering
	\begin{tabular}{lc}
		\toprule[1.5pt]
		{\heiti 输入输出} & {\heiti 描述}\\\midrule[1pt]
		$v_1$ & 旧版本源代码,待检测的对象 \\
		$v_2$ & 新版本源代码 \\
%		$v_3$ & 将补丁$p_2$应用于版本$v_2$后的源代码 \\
		$v_4$ & 将补丁$p_2$中的变更应用于版本$v_1$后的源代码 \\
		$p_2$ & 适用于$v_2$的补丁,待检测的对象 \\
		$s_1$ & $diff(v_2,v_1)$对版本$v_2$的影响域 \\
		$s_2$ & $diff(v_2,v_4)$对版本$v_2$的影响域 \\
		结果 & 是否冲突 \\
		\bottomrule[1.5pt]
	\end{tabular}
\end{table}

因此,实现该补丁兼容性检测方法的工具可以设计成如图\ref {structure}所示的组合形式。

\begin{figure}[H]
	\centering
	\includegraphics[height=.5\columnwidth]{chap03_structure}
	\caption {整体架构}	
	\label {structure}
\end{figure}

可见,兼容性检测工具可以拆分成三个模块:
\begin{itemize}
	\item 补丁版本迁移模块:实现解决方案中的补丁应用步骤。
	\item 影响域分析模块:实现解决方案中的软件变更影响域分析过程。
	\begin{itemize}
		\item 差异性分析模块:实现$diff$函数。
		\item 影响分析模块:实现$impact$函数。
	\end{itemize}
	\item 冲突判定模块:实现解决方案中的软件变更冲突检测方法。
\end{itemize}

整个工具的运作流程可以参考图\ref {solution_all}。其输入输出可以参考表\ref {all_io2}。

\begin{figure}[H]
	\centering
	\includegraphics[width=.8\columnwidth]{chap03_all}
	\caption {工具流程}
	\label {solution_all}	
\end{figure}

\begin{table}[H]
	\caption{输入输出对照表}
	\label{all_io2}
	\centering
	\begin{tabular}{lc}
		\toprule[1.5pt]
		{\heiti 输入输出} & {\heiti 描述}\\\midrule[1pt]
		$v_1$ & 旧版本源代码,待检测的对象 \\
		$v_2$ & 新版本源代码 \\
		$v_4$ & 将补丁$p_2$中的变更应用于版本$v_1$后的源代码 \\
		$p_2$ & 适用于$v_2$的补丁,待检测的对象 \\
		$p_1$ & 补丁,版本$v_1$和$v_2$之间的变更集合\\
		$p_3$ & 补丁,版本$v_4$和$v_2$之间的变更集合\\
		$s_1$ & 变更集合$p_1$对版本$v_2$的影响域 \\
		$s_2$ & 变更集合$p_3$对版本$v_2$的影响域 \\
		结果 & 是否冲突\\
		\bottomrule[1.5pt]
	\end{tabular}
\end{table}

其工作流程可以简述如下:

\begin{enumerate}
	\item 采用版本控制系统进行代码版本管理,并进行补丁版本迁移,得到应用于新版本的补丁后代码版本$v_4$。
	\item 根据得到的三个版本代码$v_1$、$v_2$、$v_4$,分别分析其程序间差异性,生成对应的程序变更集合。
	\item 根据得到的程序变更集合,进行不同版本间的变更影响分析,生成语义影响域。
	\item 根据得到的语义影响域,进行冲突分析。
\end{enumerate}

\subsection{方法概述}

%如章节\ref {patch_app}中所述,我们可以采用常见的版本控制工具提供的三路归并算法进行补丁的版本迁移工作。具体来讲,其流程可以简述如下:
%\begin{enumerate}
%	\item 将补丁$p_1$应用到版本$v_1$,获得旧版本应用补丁后的代码,即其版本为$v_3 = patch(p_1,v_1)$。
%	\item 采用三路归并算法实现$v_4 = merge(v_2,v_3)$过程,获得新版本应用补丁后的代码$v_4$。
%	\item 解决分支合并中可能出现的冲突问题。
%\end{enumerate}

%其中,通过三路归并算法进行的合并工作中可能会出现冲突,这说明版本$v_2$和$v_3$之间存在语法冲突,即这两个版本在语法层面上不兼容。我们可以通过人工修改的方式进行修复,实现语法层面上的兼容性。

本节中主要对补丁兼容性检测方法进行概述,并说明方法中各步骤的目的。

如前所述,兼容性检测方法的第一步是进行补丁版本迁移,该过程即为$merge$函数的具体实现,它是为了将专为某个特定版本而设计的补丁应用于其他版本,并且消除该过程中可能产生的语法错误。

该步骤对应检测工具中的版本迁移模块,在实现中采用git工具的三路归并算法完成。具体来讲,其流程可以简述如下:
\begin{enumerate}
	\item 将补丁$p_1$应用到版本$v_1$,获得旧版本应用补丁后的代码,即其版本为$v_3 = patch(p_1,v_1)$。
	\item 采用三路归并算法实现$v_4 = merge(v_2,v_3)$过程,获得新版本应用补丁后的代码$v_4$。
	\item 解决分支合并中可能出现的冲突问题。
\end{enumerate}

其中,通过三路归并算法进行的合并工作中可能会出现冲突,这说明版本$v_2$和$v_3$之间存在语法冲突,即这两个版本在语法层面上不兼容。在该模块的具体实现中,我们可以使用第三方的合并工具Beyond Compare解决冲突,以实现语法层面上的兼容性。

%该补丁应用的过程主要用于解决补丁的版本适应性问题,整个过程可以用算法\ref {algo_patch}描述如下。
%
%\begin{algorithm}[H]
%	\caption{补丁应用}
%	\label{algo_patch}
%	\begin{algorithmic}[1]
%		\Require $v_1,v_2,p_1$
%		\Ensure $v_4$
%		\State $v_3 \gets patch(p_1,v_1)$
%		\State $v_4 \gets merge(v_2,v_3)$
%		\State $v_4 \gets resolve(v_4)$
%		\State\Return{$v_4$}
%		\State
%		\Function {merge}{$new,old\_patched$} 
%			\State\Return{$3-way-merge(old,new,old_patched)$}
%		\EndFunction
%		\State
%		\Function {resolve}{$version$} 
%			\State\Return{$mergetool(version)$}
%		\EndFunction		
%	\end{algorithmic}
%\end{algorithm}

%\subsection{语义影响域分析}
%\label {sia}

检测方法的第二个步骤是软件变更影响域分析的过程。通过该过程,我们能够找到两个补丁中的变更集合所对应的影响域,该影响域划定了变更集合对于其他程序代码结构的影响范围。

实际上,语义影响域分析主要分为两个分析过程,即程序间差异分析和变更影响分析,通过这两个分析的协作来完成整个语义影响域的分析。这两个子过程即为$diff$和$impact$函数的具体实现。

变更影响域分析所对应的工具模块是影响域分析模块,该模块在实际实现过程中也同样划分为两个子模块:
\begin{itemize}
	\item 差异性分析模块:该模块采用ASTro工具来实现具体的程序差异性分析算法。
	\item 影响分析模块:该模块采用jpf-regression工具来实现具体的变更影响分析算法。
\end{itemize}

在具体的模块实现过程中,我们发现一些主要需要解决的问题包括:
\begin{enumerate}
	\item 修正ASTro的分析结果,以提高精确度。
	\item 改进jpf-regression工具,包括:
	\begin{enumerate}
		\item 修复工具中自带的Bug。
		\item 修改工具的分析过程。
		\item 增加影响追踪系统。
		\item 增加错误记录系统。
	\end{enumerate}
	
	\item 实现实验过程的批量化和自动化。
\end{enumerate}

详细的影响域分析方法介绍与实现将在第\ref {chap_impact}章中给出。

%
%\subsubsection{程序差异性分析}
%
%程序差异性分析是$diff$函数的实现。它主要用于分析两个不同版本的程序间的差异性,其结果即我们所需要的程序变更集合。近年来程序差异性分析方面有不少工作,实现了一些较为成熟的比较工具,我们可以使用这些工具来实现该分析过程。
%
%在本文的组合架构中,程序差异性分析的主要任务是接受两个不同版本的源代码,并返回代码间的结构化差异信息。结构化的差异信息可以视为补丁的一种,只不过它具有比常见的采用Unix diff工具生成的$.patch$类型的补丁文件更丰富的信息,能够描述以程序语法结构的形式对软件变更进行描述。
%
%该分析过程应该满足如下需要:
%\begin{itemize}
%	\item 输入为两个不同版本的代码。
%	\item 输出为源代码间的软件变更集合。
%	\item 每条变更描述语句或基本块级别的变更。
%	\item 每条变更描述新旧程序结构的相关信息和其关联关系。
%	\item 每条变更描述了其所属的作用域。
%\end{itemize}
%
%选择这样的分析结果类型是为了后续分析过程的方便,因为变更影响分析需要我们提供软件变更集合作为输入,而$.patch$类型的补丁文件只描述文本行的变更,不包含语法信息,我们无法从中提取出所需的语法层面的变更信息。
%
%一种比较好的选择是采用AST差异性分析,因为抽象语法树中包括了足够多的语法结构信息。
%
%
%
%%图中所描述的软件变更集合可以定义如下:
%%\begin{definition}
%%	$ change\_set = \{ (old_i,new_i) \mid  old_i \subset Structure,new_i \subset Structure, i \subset \mathbb{N} \}$
%%\end{definition}
%
%
%
%\subsubsection{变更影响分析}
%
%程序变更影响分析是$impact$函数的具体实现。主要用于获取变更集合对其他程序结构的影响域,该集合所包含的程序结构直接或间接地受到变更集合中的元素影响。近年来这方面比较成熟的工作也有不少,因而可以直接选择合适的变更影响分析算法作为该过分析过程的具体实现。
%
%在本文的组合架构中,该分析过程应当接受两个不同版本代码间的变更集合作为输入,并输出变更集合所对应的影响集合,这也就是我们所需要的变更集合的语义影响域。变更影响分析的过程可以通过控制流、数据流等信息分析出变更集合中每条变更对其他程序结构是否存在影响,并进行闭包计算即可。
%
%本文对于该分析过程的要求如下:
%\begin{itemize}
%	\item 接受两个不同版本代码间的变更集合作为输入。
%	\item 计算得到该变更集合所对应的影响集合。
%	\item 计算过程中可以指定受影响的范围和元素类型。
%	\item 影响集合中的元素按照其不同类型进行分类。
%	\item 具有影响追踪系统,将计算影响域的过程进行记录。方便后续的冲突分析过程进行回溯。
%\end{itemize}
%
%其中影响追踪系统可以定义成如下类型的函数。该函数接受一个影响域分析函数$ia(v_i,v_j)$,并返回对应的依赖关系集合。
%
%\begin{definition}
%	$impact\_track(ia(v_i,v_j)):(Code \times Code \mapsto {Structure}) \mapsto {depend}$
%\end{definition}
%
%
%%该分析过程如图\ref {impact_analyzer}所述,其中的$impact$函数可以任意选择某一能够满足上述要求的变更影响分析算法。
%%
%%图中所描述的变更影响集合可以定义如下:
%%\begin{definition}
%%	$impact\_set = \{ (structure_i) \mid  structure_i \subset Structure, i \subset \mathbb{N}\}$
%%\end{definition}
%%
%%\begin{figure}[H]
%%	\centering
%%	\includegraphics{chap03_impact}
%%	\caption {程序间差异分析}
%%	\label {impact_analyzer}	
%%\end{figure}
%
%在完成了上述两项子分析过程后,我们就完成了整个语义影响域分析过程,并获得了两个不同版本代码的变更集合和对应的其语义影响范围。接下来就可以进行具体的兼容性分析工作。
%
%整个语义影响域分析过程可以用算法\ref {algo_impact}描述。
%
%\begin{algorithm}[H]
%	\caption{语义影响域分析算法}
%	\label{algo_impact}
%	\begin{algorithmic}[1]
%		\Require $v_i,v_j$	
%		\Ensure $s$
%		\Function {ia}{$v_i,v_j$}
%			\State $p \gets diff(v_i,v_j)$
%			\State $s \gets impact(p,v_i)$
%			\State\Return{$s$}
%		\EndFunction
%	\end{algorithmic}
%\end{algorithm}

%\subsection{冲突分析}

检测方法的第三步为软件变更冲突检测方法,该过程即为$conflict$函数的具体实现。我们将对比两个补丁的变更影响域,判定他们是否重叠,并以此为依据找到可能发生冲突的代码位置。更精确的冲突判定过程需要人工分析的辅助。

冲突检测方法对应工具中的冲突判定模块,该模块实现了具体的冲突检测算法,它将接收影响域分析模块的输出,并分析并找到变更影响域间的重叠。

详细的冲突检测方法介绍与实现可以参考第\ref {chap_conflict}章。
%
%理论上来讲,这种简单比对即可发现两个版本间的兼容性问题,因为一旦发生重叠,那么重叠的代码部分显然是会发生兼容性问题的。然而在实际情况中,受限于工具的精度,我们往往不能达到理论上的准确度,而可能会发生误报(False Negative)等情况。
%
%显然,如果重叠不存在,则兼容性是可以得到满足的。而对于如何界定重叠部分的兼容性,则需要人工分析的介入,因为这部分代码的兼容性与补丁的功能目标密切相关,而我们无法从代码中获取到这种信息,因而只有依靠外界来提供,并以此为依据进行详细分析过程,界定这部分代码是否真的存在冲突。
%
%在进行人工分析的过程中,我们不仅需要知道哪部分代码出现了重叠现象,而且还需要知道是哪些变更影响到了这部分代码,因而我们需要引入影响追踪系统来记录变更影响分析过程的轨迹。影响追踪系统可以记录下变更影响分析过程中的每一步,从而获取到程序结构间的影响关系链,事后通过回溯即可追踪到具体的软件变更可见,本过程中主要需要影响追踪系统的回溯模块的支持。

%可以看得出来,由于使用了更严格的冲突定义,我们的方法会造成一定的过高估计的结果。

%该分析过程可以参考算法\ref {algo_compatible}。
%
%\begin{algorithm}[H]
%	\caption{冲突分析}
%	\label{algo_compatible}
%	\begin{algorithmic}[1]
%		\Require $s_1=ia(v_2,v_1)$,$s_2=ia(v_2,v_4)$\\
%				 \quad \quad $t_1=impact\_track(ia(v_2,v_1))$, $t_2=impact\_track(ia(v_2,v_4))$
%		\Ensure $isCompatible(diff(v_2,v_1), diff(v_2,v_4))$
%		\If{$s_1 = \varnothing \lor s_2 = \varnothing$}
%			\State $result \gets True$
%		\Else
%			\State $s \gets s_1 \cap s_2$
%			\If{$s = \varnothing$}
%				\State $result \gets True$
%			\Else				
%				\State $result \gets Manual\_analysis(s_1, s_2, t_1, t_2)$
%			\EndIf 
%		\EndIf
%		\Return $result$
%	\end{algorithmic}
%\end{algorithm}

%整个分析过程的架构如图\ref {isCompatible}所示。
%
%\begin{figure}
%	\centering
%	\includegraphics{chap03_isCompatible}
%	\caption {兼容性分析}
%	\label {isCompatible}	
%\end{figure}


\section{本章小结}
本章主要介绍了软件补丁兼容性检测所关注的主要问题以及如何解决该问题。

章节\ref {define_problem}中对兼容性问题进行了定义和分析。

章节\ref {problem_solve}根据将前一节中提到的分析方法,提出了一套通用的兼容性检测方法和对应的工具实现,并简要介绍了其中涉及的方法。
%\chapter{方法实现}
本章中将主要介绍如何将补丁版本迁移、语义影响范围分析、兼容性分析等过程具体实现之,并组合成整体流程。

\section{补丁版本迁移}

如前所述,补丁版本迁移主要用于解决问题\ref {patch_reversion}。我们在实现该过程时,主要采用了git工具和Beyond Compare工具来实现版本合并和冲突解决的过程。

我们下面将利用git的分支功能进行版本合并。首先介绍一些前提假设:

\begin{enumerate}
	\item 假设git的工作目录如图\ref {git_work_dir}所示。其中git\_working\_dir为git目录,下属目录包括源代码仓库src和git目录的隐藏文件夹.git。另外与git目录平级的目录patch下面则存储了具体的补丁文件。

	\item 假设补丁$p_1 = old.patch$,并且采用git diff或其他Unix diff命令生成。
\end{enumerate}

\begin{figure}[H]
	\centering
	\includegraphics{chap04_work_dir}
	\caption {git工作目录}
	\label {git_work_dir}	
\end{figure}

下面将具体介绍整个版本合并的过程。


首先在工作目录下将git切换到主分支master。然后在主分支master中提交代码版本$v_1$。

\begin{lstlisting} [style=BashInputStyle]
git checkout master
git add .
git commit -a -m "old version committed"
\end{lstlisting}

然后新建并切换到分支new中,并提交代码版本$v_2$。
\begin{lstlisting} [style=BashInputStyle]
git checkout -b new
git add .
git commit -a -m "new version committed"
\end{lstlisting}

再次从主分支新建并切换到分支patch,然后将补丁$p_1$应用到版本$v_1$,获得旧版本应用补丁后的代码,其版本为$v_3$,并提交代码版本$v_3$。此时我们所应用的补丁$p_1$是专为版本$v_1$设计,所以应用时不会出现问题。

\begin{lstlisting} [style=BashInputStyle]
git checkout master
git checkout -b patch
git apply ../patch/old.patch
git add .
git commit -a -m "patched version from old version committed"
\end{lstlisting}

然后再切换回分支new,将分支patch合并入分支new,获得新版本应用补丁后的代码,其版本为$v_4$,并使用Beyond Compare工具解决可能出现的冲突问题。将冲突解决完毕后,再提交版本$v_4$。

\begin{lstlisting} [style=BashInputStyle]
git checkout new
git merge patch
git mergetool
git commit -a -m "patched version from new version committed"
\end{lstlisting}

如果确实有冲突,那么git mergetool命令会调用第三方的可视化合并工具并引导你去解决冲突。这里我们采用的合并工具即为Beyond Compare,它会展开一个可视化的界面,并给出冲突位置的提示,方便进行人工选择、合并。

整个过程可以参考图\ref {git_merge}。

\begin{figure}[H]
	\centering
	\includegraphics{chap04_git_merge}
	\caption {git版本合并}
	\label {git_merge}	
\end{figure}

\section{语义影响范围分析}

如前所述,语义影响范围分析主要用于解决问题\ref {impacted_area}。语义影响范围分析需要首先定义好影响范围和影响元素的级别。在实际情况中,我们主要采用了较为简单的分析过程,即将影响范围限制在方法内部,但是将受影响元素设置为程序语句级别,以保证精度。

由于我们采用了现有工具来完成具体的程序间差异分析和变更影响分析过程,因而这两个子过程的分析算法已经无需我们自己实现。我们可以将主要精力放在如何整合两个子过程的工作上,并将其工具转化为适用于本问题的具体情况。

在具体的实现过程中,我们发现一些主要需要解决的问题包括:
\begin{enumerate}
	\item 修正ASTro的分析结果,以提高精确度。
	\item 改进jpf-regression工具,包括:
		\begin{enumerate}
			\item 修复工具中自带的Bug。
			\item 修改工具的分析过程。
			\item 增加影响追踪系统。
			\item 增加错误记录系统。
		\end{enumerate}
	
	\item 实现实验过程的批量化和自动化。
\end{enumerate}

\subsection{程序间差异分析}

如前所述,我们在实际工作中采用了ASTro工具来完成具体的程序间差异分析过程,它专为Java语言设计。在实际使用中,为了满足我们的需要,我们进行了一些改进。

首先,我们采用了shell脚本完成了分析过程的批量化和自动化,能够自动对整个软件系统的所有代码进行批量化处理,并循环调用ASTro进行分析。若需要对不同的代码进行修改,只需要修改对应的实验数据存放位置和其代码所依赖的JAR包即可。在这部分工作中,脚本代码主要完成了以下内容:

\begin{itemize}
	\item 实验数据定位,包括Java源代码和编译后的Class文件等。
	\item 根据代码的存放路径,计算其对应Class文件的位置。
	\item 获取代码文件名,以确定本次分析的对象。
	\item 实验数据的依赖JAR包定位。
	\item 创建输出文件目录。
	\item 定义ASTro的输入参数,包括输入文件位置、输出文件位置、查找路径等。
	\item 调用ASTro进行单次分析。
\end{itemize}

其中ASTro工具的使用格式可参考如下,其具体各参数的定义参考表\ref {ASTro}。

\begin{lstlisting} [style=BashInputStyle]
ASTDiffer 3/27/2013
USAGE: java ASTDiffer -original <file>.java -modified <file>.java 
-dir <output folder>
OPTIONAL: -file <fileName> -ocp <classpath> -mcp <classpath> 
-oco <outputDir> -mco <outputDir> -cs -xml
\end{lstlisting}	

\begin{table}
	\caption{ASTro参数对照表}
	\label{ASTro}
	\centering
	\begin{tabular}{llc}
		\toprule[1.5pt] 
		{\heiti 参数名} & {\heiti 描述} & {\heiti 启用}\\\midrule[1pt]
		-file & 分析目标的名字 & 是 \\
		-dir & 输出路径 & 是 \\
		-ocp & 旧版本代码的Classpath & 是\\
		-mcp & 旧版本代码的Classpath & 是\\
		-original    & 旧版本代码的位置 & 是\\
		-modified   & 新版本代码的位置 & 是\\
		-xml   & 以XML格式输出结果 & 是\\
		-cs   & 以变更脚本(Change Script)格式输出结果 & 否\\
		-heu   & 以启发式的方式进行匹配 & 是\\
		\bottomrule[1.5pt]
	\end{tabular}	
\end{table}

其次,我们采用了shell脚本完成了对后续分析过程的支持,能够自动批量化创建变更影响分析所需的配置文件。配置文件为自定义的JPF格式,通过类似键值对的方式定义了各项属性的值。JPF文件的格式等来自于Java Path Finder框架的设计,是该框架运行所必须的配置文件。该配置文件的具体属性为自定义设置,可以参考表\ref {JPF_prop}所述。

\begin{table}
	\caption{JPF属性对照表}
	\label{JPF_prop}
	\centering
	    \begin{tabular*}{\linewidth}{lp{10cm}}
	    	\toprule[1.5pt]
	    	{\heiti 属性名} & {\heiti 描述} \\\midrule[1pt]
	    	target & 分析的目标 \\
	    	sourcepath & 源代码路径\\
	    	rse.ASTResults & ASTro工具的输出文件位置\\
	    	rse.newClass & 新版本代码的Class文件位置\\
	    	rse.oldClass    & 旧版本代码的Class文件位置\\
	    	rse.dotFile   & jpf-regression工具的Dot格式输出文件位置\\
	    	\bottomrule[1.5pt]
	    \end{tabular*}
\end{table}

在使用shell脚本调用ASTro工具进行分析和输出变更影响分析的配置文件时,考虑到实际使用中,我们需要将新版本$v_2$作为对比的基准,以获取一致的行号。因而在进行变更影响分析时,我们需要进行相应配置,使得:
\begin{itemize}
	\item $s_1 = impact(diff(v_2,v_1),v_2)$,求得补丁$p_1 = diff(v_2,v_1)$对版本$v_2$的影响范围$s_1$。
	\item $s_2 = impact(diff(v_2,v_4),v_2)$,求得补丁$p_2 = diff(v_2,v_4)$对版本$v_2$的影响范围$s_2$。
\end{itemize}

实际上,也就是说:
\begin{itemize}
	\item 补丁$p_1 = diff(v_2,v_1)$,即将新版本$v_2$视为“旧版本”,将旧版本$v_1$视为“新版本”。
	\item 补丁$p_2 = diff(v_2,v_4)$,即将新版本$v_2$视为“旧版本”,将新版本应用补丁后的版本$v_4$视为“新版本”。
\end{itemize}

在实际操作中,我们只需做这样的版本交换即可。

受限于ASTro工具的具体实现,其输出结果的存在一定的问题,主要包括:
\begin{enumerate}
	\item 对某些代码文件无法完成差异性分析。
	\item 对某些代码文件输出结果不准确,存在过高估计(over-estimate)的问题。
\end{enumerate}

对于第一个问题,由于无法知道该工具的源代码,我们无法解决,不过这只是极少数现象。

对于第二个问题,我们分析其结果可以发现,其结果中存在误报的情况,即某些代码行并未发生变更,然而工具却报告其发生了诸如移动、先删后增之类的伪变更。同样由于无法知道该工具的源代码,我们无法从算法的角度进行修改,不过对于这样的情况,我们可以对其输出结果进行一定的预处理,将这些误报的情况进行过滤,保留一个真变更子集合即可。

预处理算法可以用伪代码\ref {xml}进行描述:

\begin{algorithm}
	\caption{XML结果过滤算法}
	\label{xml}
	\begin{algorithmic}[1]
		  \REQUIRE $n \geq 0 \vee x \neq 0$
		  \ENSURE $y = x^n$
		  \STATE $y \gets 1$
		  \IF{$n < 0$}
		  \STATE $X \gets 1 / x$
		  \STATE $N \gets -n$
		  \ELSE
		  \STATE $X \gets x$
		  \STATE $N \gets n$
		  \ENDIF
		  \WHILE{$N \neq 0$}
		  \IF{$N$ is even}
		  \STATE $X \gets X \times X$
		  \STATE $N \gets N / 2$
		  \ELSE[$N$ is odd]
		  \STATE $y \gets y \times X$
		  \STATE $N \gets N - 1$
		  \ENDIF
		  \ENDWHILE
	\end{algorithmic}
\end{algorithm}

我们可以归纳证明这种预处理操作的正确性。由于变更对于代码的影响是链式的,对于某次变更影响分析的结果集合$s_{i,j} = ia(v_i,v_j)$而言,假设对于其中任意一个受影响的元素$e_k$,其中$k \subset \mathbb{N}$,其影响来源可能包括如下几种可能:
\begin{enumerate}
	\item 其影响仅来源于变更$c_1$。
		\begin{itemize}
			\item 如果$c_1$为真变更,那么删除所有伪变更对于$e_k$没有影响。
			\item 如果$c_2$为伪变更,那么删除所有伪变更会导致$e_k$从集合$s_{i,j}$中被删除,但此时$e_k$本身即为伪影响,集合$s_{i,j}$的正确性会得到提高。
		\end{itemize}
	\item 其影响来源于多条变更$c_1,c_2\dots,c_m$,其中$m \subset \mathbb{N}$。
		\begin{itemize}
			\item 假若所有变更均为真变更,那么删除所有伪变更对于$e_k$没有影响。
			\item 假若来源变更集合中包括某几条伪变更,那么删除所有伪变更之后,仍然存在其他真变更,这些真变更仍然会在变更影响分析中导致$e_k$被添加到集合$s_{i,j}$中,因而也不会使集合$s_{i,j}$的正确性下降。
			\item 假若所有变更均为伪变更,那么删除所有伪变更会导致$e_k$从集合$s_{i,j}$中被删除,但此时$e_k$本身即为伪影响,集合$s_{i,j}$的正确性会得到提高。
		\end{itemize}
\end{enumerate}

可见,我们的预处理操作是正确的,它不会导致结果集合$s_{i,j} = ia(v_i,v_j)$的正确性降低。

整个程序间差异性分析过程可以参考图\ref {diff}。

\begin{figure}[H]
	\centering
%	\includegraphics{}
	\caption {程序间差异性分析流程}
	\label {diff}	
\end{figure}

\subsection{变更影响分析}

我们在变更影响分析过程中采用了jpf-regression工具来进行具体的影响分析过程。然而jpf-regression工具中变更影响分析只是其中的一个子模块,主要用于为其后续的DiSE分析过程服务。因而在实践过程中,我们采用的解决办法是重用jpf-regression的代码,并对其加以改造,主要的变化包括:
\begin{enumerate}
	\item 修改分析流程。
	\item 增加影响追踪系统。
	\item 增加错误记录系统。
	\item 使其适应大规模批量化分析的需要。
	\item 已知Bug修复。
\end{enumerate}

下面分别进行介绍。

\subsubsection{分析流程}
jpf-regression作为Java Path Finder框架的一个插件,事实上在使用时需要遵守该框架的约束,有明确的执行流程规定。然而在实际使用中,该流程约定与我们的实际情况并不合用,因而我们对此进行了一定的修正。

事实上,原执行流程约定,每次分析以源代码文件中的Main函数作为入口,探索并分析Main函数所调用的其他函数。该流程对于大部分情况而言是具有实际意义的,并且由于只考虑Main函数及其调用的函数,工具可以节约分析的开销,更快的得出结论,而忽略掉其他事实上并未在执行过程中被涉及到的函数。

然而对于我们的分析需要而言,该流程只能覆盖到部分情况,对于其他类型的软件系统而言可能并不适用,例如以Eclipse JDT Core项目而言,该项目主要用于为Eclipse软件系统的其他组件提供服务,因而在实际中以JAR包的形式作为库函数而存在。对于这类以库函数形式对外提供服务的源代码而言,他们并不存在入口函数,也无法预知到底会有哪些函数会被外界所调用。因而对于这类情况而言,我们需要在分析过程中覆盖其所有函数,以保证结果的完整性和正确性。

我们对于流程的修改可以参考图\ref {impact_process}进行对比。

\begin{figure}[H]
	\centering
%	\includegraphics{}
	\caption {变更影响分析流程对比}
	\label {impact_process}	
\end{figure}

\subsubsection{影响追踪}

在后续的兼容性分析过程中,对于得到的冲突结果,我们需要对其追根溯源,挖掘其起始的影响来源,找到对应的影响来源变更集合,以进行人工分析对比,判定该情况是否确实冲突。

因此,我们需要在变更影响分析的阶段也加入影响追踪系统的记录模块,以便记录下变更影响的轨迹,根据这些信息为后续的分析过程提供便利。

【可以介绍下相关的类设计】

\subsubsection{错误记录}

原有的jpf-regression工具由于是单次分析过程,因而一旦在运行过程中遇到问题,就会采用抛出异常终止运行的方式结束分析。然而我们在实际情况中需要进行大规模的分析作业,如果仅仅在其中单个文件的分析过程中出错就终止整个分析作业,会造成极大的时间和计算资源的浪费。

因而我们对该工具的异常处理方式进行了修改,使其在单次分析过程中如果遇到问题,则会及时抛出异常,但并不终止整个程序的运行,而采取了继续往下执行并分析其他文件的策略。然而分析错误是确实存在的,为了不丢失这类错误信息,我们为工具添加了错误记录系统,不断记录单次分析过程中遇到的问题。

我们对程序运行过程中可能出现的报错情况进行了分类,并设计了专门的错误统计类以专门别类的对错误情况进行记录和统计。

【可以介绍下相关的类设计】

\subsubsection{大规模分析}

原有的jpf-regression工具只能支持单次分析过程,在实际情况中我们需要工具具备大规模批量化分析的能力,以应对大规模软件系统的实际需求。为此我们可以保留原有的单次分析过程,然后在上层进行封装,循环多次调用单次分析过程,以达到批量化自动分析的效果。这个过程由于进行了封装,对于用户而言是透明的。

同时,在进行大规模分析的时候,输出文件的命名格式也需要修改。在原单次分析过程中,输出文件直接采用被分析的方法名进行命名。对于分析小型文件而言,这种设计就足够了,然而在大规模分析的时候,我们需要进行一定的优化。

由于大规模分析时,可能存在一些现象,例如:
\begin{enumerate}
	\item 函数重载
	\item 不同版本间的代码其方法可能无法一一匹配。例如有的方法仅在单个版本中出现。
\end{enumerate}

在这种情况下,我们采用的命名格式为$UniqMethodName+HashCode(UniqMethodName)+ExtensionName$。其中,$UniqMethodName$由方法名、方法参数类型、方法返回类型三组成,保证方法名的唯一性。再利用Java中的HashCode方法,对$UniqMethodName$计算其HashCode作为其后缀。最后$ExtentionName$即为文件扩展名,在jpf-regression中$ExtentionName = “.dot”$。

同时我们也保留了原有的单次分析能力,以满足实际情况中的其他需要,例如进行小规模的案例分析。

其次,在进行大规模分析的情况下,由于实验数据量的庞大,我们无法按照单次分析过程中那样手动查看并分析实验结果。因此,为了适应这种需要,我们还增加了数据统计模块,使得程序具备一定的自动化分析实验结果的能力。


【可以介绍下相关的类设计】

\subsubsection{Bug修复}

在实际使用jpf-regression进行实验的过程中,我们发现该工具存在一些Bug,这些Bug或多或少的导致了分析结果的正确性和精度降低。,我们对其中力所能及的Bug进行了修复,并对这些Bug进行了总结。

目前发现的已知Bug包括:
\begin{enumerate}
	\item 内部类无法匹配方法名,忽略
	\item 只有单个文件存在该方法时无法匹配方法名,已修复
	\item 映射$CFG_{old}$时判断条件出错,已修复
\end{enumerate}

存在的已知问题包括:
\begin{enumerate}
	\item 有的文件会出现搜索错误,jpf-regression工具所依赖的其他JPF框架插件的问题,忽略
	\item XML文件无法找到并装载,ASTro工具的问题,忽略
\end{enumerate}
\section{兼容性分析}
\chapter{兼容性分析算法}
介绍具体的兼容性算法,如果两个影响集合存在交集,是否确实是不兼容的。

\chapter{软件变更冲突检测方法}
\label {chap_conflict}

本章中主要介绍软件变更冲突检测方法,包括具体的检测算法、相应的模块设计与实现等。

软件变更冲突检测主要用于对变更影响域之间是否发生冲突进行判定。本文中实现了简单的自动分析算法,通过判断变更影响域之间是否存在重叠来找到可能存在冲突的代码位置,并结合人工分析来完成最后的判定。

下面对该冲突检测方法和其对应模块的设计与实现进行具体的介绍。

\section{相关定义}

\label {conflict_define}

为了进行变更的冲突检测,需要知道什么是冲突。下文中的冲突都指语义冲突。

根据前文中的讨论,冲突实际上是指两个补丁所对应的不同变更集合之间存在某两条互斥的变更,即这两条变更的影响域发生了重叠。因此,冲突可以较形式化的定义如下,其中$\mathcal{C}$是某个版本的源代码文件中按照该语言的合法语法结构组织起来的代码,$\mathcal{P}$是变更集合,$impact$是求解影响域的函数:

\begin{definition}
	$conflict:\mathcal{P} \times \mathcal{P} \mapsto \{T,F\}$。$\forall p_i,p_j \subset \mathcal{P}$,如果$impact(p_i,v) \cap impact(p_j,v) \neq \varnothing$,那么就有$conflict(p_i,p_j) = T$,其中$v \subset \mathcal{C},i,j \subset \mathbb{N}$。即如果两个补丁分别对应的变更集合在某相同版本代码上的变更影响域之间产生了交集,那么就认为补丁间发生了冲突。
\end{definition}

综上所述,冲突是由于来自两个变更集合的某两条不同变更分别直接或间接的影响到了某个相同的语法结构而造成的,可见$conflict$函数描述了变更冲突检测的过程,它接受两个变更集合作为输入,并根据变更影响域计算其是否发生了冲突。

\section{分析方法}
\label {chap_conflict}

如前所述,冲突检测的过程即为$conflict$函数的具体实现。该分析过程将对比两个补丁间的变更影响域,通过判断变更影响域间是否重叠来检测其兼容性。

%该函数的实现可以定义如下:
%
%\begin{definition}
%		如果$\exists c_m \subset p_i, \exists c_n \subset p_j$,其中$c_m = change(structure_m),c_n = change(structure_n)$, $structure_m \subset Struct(v_k), structure_n \subset Struct(v_k), v_k \subset Code$,并且$i,j,k,m,n \subset \mathbb{N}$。
%		那么有$conflict(p_i,p_j) == true \iff (structure_m == structure_n) \lor (impact(p_i,v_k) \cap impact(p_j,v_k) \neq \varnothing)$,其中$p_i,p_j \subset Pat(v_k)$。
%\end{definition}

理论上来讲,由于重叠部分的代码显然是可能会发生冲突的,这种简单比对即可发现两个补丁间的兼容性问题。然而在具体实现中,受限于工具的精度,冲突检测方法往往不能达到理论上的准确度,而可能产生误报等情况。

如果重叠不存在,则补丁间的语义没有发生相互覆盖,因而不存在冲突。如果存在重叠,则对于重叠部分的代码而言,判断该位置是否确实发生了冲突需要人工分析的辅助。因为该代码位置的冲突判定与补丁的变更目的密切相关,而代码中无法直接获取到这种信息,因此只有依靠外界的干预来判定这部分代码是否确实存在冲突。

人工分析过程不仅需要知道哪部分代码出现了重叠,还需要知道是哪些变更影响到了这部分代码。因此冲突检测过程需要具有影响回溯的能力。由于变更影响域分析中的影响追踪系统会记录其分析过程的轨迹并存储程序语法结构间的影响关系,因此只需在冲突检测时对重叠部分的代码进行回溯即可追踪到具体的软件变更。

%可以看得出来,由于使用了更严格的冲突定义,我们的方法会造成一定的过高估计的结果。

该冲突检测过程可以参考算法\ref {algo_compatible}。

\begin{algorithm}[H]
	\caption{冲突检测}
	\label{algo_compatible}
	\begin{algorithmic}[1]
		\Require $s_1=impact(diff(v_2,v_1),v_2)$,$s_2=impact(diff(v_2,v_4),v_2)$
%		\\
%		\quad \quad $t_1=impact\_track(ia(v_2,v_1))$, $t_2=impact\_track(ia(v_2,v_4))$
		\Ensure $conflict(diff(v_2,v_1),diff(v_2,v_4))$
		\If{$s_1 = \varnothing \lor s_2 = \varnothing$}
		\State $result \gets False$
		\Else
		\State $s \gets s_1 \cap s_2$
		\If{$s = \varnothing$}
		\State $result \gets False$
		\Else				
		\State $result \gets Manual\_analysis(s_1, s_2)$
		\EndIf 
		\EndIf \\
		\Return $result$
	\end{algorithmic}
\end{algorithm}


\section{模块设计与实现}
\label {chap_mod}

冲突判定模块主要需要实现$conflict$函数的实际功能。目前根据上文所述的冲突检测算法实现了较为简单的自动分析过程,更精确的分析结果需要人工分析过程的辅助。

%\subsection{设计}
%
%在该模块的设计中,其输入输出过程可以描述如图\ref {com},输入输出的具体描述参见表\ref {com_io}。
%
%该模块的核心在于冲突分析算法\ref {algo_compatible},因此,根据算法中的描述,该模块可以设计如图\ref {des_com}所示。

考虑到该模块需要实现接受影响域分析模块的输入,并完成冲突检测的功能,该模块的核心任务应当包括:
\begin{itemize}
	\item 输入:读取软件变更影响域分析过程的输出,即该模块需要接受两个变更影响域作为其输入。
	\item 输出:找到可能发生冲突的代码位置,并输出其影响来源。即该模块需要输出变更影响域重叠处的代码位置(即可能发生冲突的位置)及其相关的影响关系。
	\item 影响域计算:存储读取的影响域信息,并计算其是否发生重叠。即该模块需要实现上文提出的冲突检测算法。
%	\item 冲突分析:根据影响域重叠,对找到的冲突进行影响回溯。
\end{itemize}


%因此,在冲突判定模块中,其流程可以设计如下:
%\begin{enumerate}
%	\item 读取影响域分析模块的结果
%	\item 计算是否发生影响域重叠
%	\item 对于发生了重叠现象的影响域,判定为可能发生冲突
%	\item 回溯冲突代码的影响来源并输出其影响关系
%	\item 根据得到的影响依赖关系进行人工分析,判定是否确实冲突
%\end{enumerate}

%\begin{figure}[H]
%	\centering
%	\includegraphics[height=.6\columnwidth]{chap03_com}
%	\caption {输入输出}
%	\label {com}	
%\end{figure}
%
%\begin{table}[H]
%	\caption{输入输出对照表}
%	\label{com_io}
%	\centering
%	\begin{tabular}{lc}
%		\toprule[1.5pt]
%		{\heiti 输入输出} & {\heiti 描述} \\\midrule[1pt]
%		$ia(v_2,v_1)$ & 影响域分析模块的输出,语义影响域 \\
%		$ia(v_2,v_4)$ & 影响域分析模块的输出,语义影响域 \\
%		输出 & 是否发生语义冲突\\
%		\bottomrule[1.5pt]
%	\end{tabular}
%\end{table}
%
%\begin{figure}[H]
%	\centering
%	\includegraphics[width=.8\columnwidth]{chap04_com_internal}
%	\caption {模块设计}
%	\label {des_com}	
%\end{figure}
%
%\subsection{实现}

该模块的实现过程主要参考了算法\ref {algo_compatible}的描述。由于其输入为影响域模块的输出,因此它在输出时可以参照其输入给出类似形式的结果,即将重叠位置和其相关的影响关系作为额外的信息输出到控制流图中,以供人工分析使用。人工分析的过程主要是参考输出的部分控制流中,被标注为可能发生冲突的节点是否确实发生了冲突。

%在实际的实现中,其输入输出可以参考表\ref {com_io2}。

%\begin{table}[H]
%	\caption{输入输出对照表}
%	\label{com_io2}
%	\centering
%	\begin{tabular}{llc}
%		\toprule[1.5pt]
%		{\heiti 输入输出} & {\heiti 描述} & {\heiti 格式}\\\midrule[1pt]
%		输入 & $diff(v_2,v_1)$对于$v_2$的语义影响域 & Dot\\
%		输入 & $diff(v_2,v_4)$对于$v_2$的语义影响域 & Dot\\
%		输出 & 语义冲突 & Dot \\
%		\bottomrule[1.5pt]
%	\end{tabular}
%\end{table}

%可见,冲突判定模块中的主要工作包括:
%\begin{itemize}
%	\item 计算影响域的重叠
%	\item 对找到的语义冲突进行影响回溯,并以Dot格式将其涉及到的部分控制流和相关的冲突信息进行输出。
%\end{itemize}

冲突判定模块中的类实现可以参考图\ref {diff}。其中impactSet类用于存储影响域,DotNode和DotEdge用于存储从Dot文件中读取到的节点和边的信息,这两个类采用了多态机制来存储节点和边的类型信息。Diff类是从Dot文件中读取影响域并计算其重叠与否的类。相关的数据结构说明参考表\ref {diff_data}。

\begin{table}[H]
	\caption{Diff数据结构}
	\label{diff_data}
	\centering
	\begin{tabular}{lllc}
		\toprule[1.5pt]
		{\heiti 数据类型} &{\heiti 数据结构} & {\heiti 用途} \\\midrule[1pt]
		static void & main(String[] args) & 实现冲突分析过程\\
		void & analyzeDot(String path, impactSet now) & 读取Dot文件,将影响域存储于now \\
		void & readFileName(String path) & 读取待分析的文件名\\
		void & deleteDot(String p) & 删除输出\\
		void & computeImpact(impactSet s,impactSet s1) & 计算重叠\\
		void & output(Set<Integer> intersection, impact now) & 输出冲突和相关控制流\\
		void & printNode(BufferedWriter writer, DotNode node...) & 写控制流节点\\
		void &  WriteEdge(Set<Integer>  impacted...) & 写控制流边\\
		\bottomrule[1.5pt] 
	\end{tabular}
\end{table}

\begin{table}[H]
	\caption{impactSet数据结构}
	\label{impact_data}
	\centering
	\begin{tabular}{lllc}
		\toprule[1.5pt]
		{\heiti 数据类型} &{\heiti 数据结构} & {\heiti 用途} \\\midrule[1pt]
		String     &  path & 该影响域对应Dot文件的位置 \\
		Set<Integer>  &  allLoc & 存储影响域中的元素 \\
		Set<DotNode>  & nodes & Dot文件中的控制流节点\\
		Set<DotEdge>  & edges & Dot文件中的控制流边\\
		Map<Integer, Set<Integer>> & lineToNodeId & 从控制流边到节点ID的映射\\
		Map<Integer, DotNode>  & idToNode & 从节点ID到节点的映射 \\
		Map<Integer, Set<DotEdge>>  &  relation & 从节点ID到边的映射 \\
		\bottomrule[1.5pt] 
	\end{tabular}
\end{table}

\begin{table}[H]
	\caption{DotNode数据结构}
	\label{node_data}
	\centering
	\begin{tabular}{lllc}
		\toprule[1.5pt]
		{\heiti 数据类型} &{\heiti 数据结构} & {\heiti 用途} \\\midrule[1pt]
		Integer    &   begin & 该基本块的起始行号 \\
		Integer    &   end  & 该基本块的结束行号 \\
		Integer    & id & 该节点的ID \\
		boolean    & changed & 该节点是否属于变更集合 \\
		\bottomrule[1.5pt] 
	\end{tabular}
\end{table}

\begin{table}[H]
	\caption{DotEdge数据结构}
	\label{edge_data}
	\centering
	\begin{tabular}{lllc}
		\toprule[1.5pt]
		{\heiti 数据类型} &{\heiti 数据结构} & {\heiti 用途} \\\midrule[1pt]
		Integer   & begin & 该边的起始节点 \\
		Integer  &   end & 该边的结束节点 \\
		\bottomrule[1.5pt] 
	\end{tabular}
\end{table}

\begin{figure}[H]
	\centering
	\includegraphics[width=.8\columnwidth]{chap04_diff}
	\caption {Diff类族}
	\label {diff}	
\end{figure}

综上所述,该模块的实际工作流程可以参考图\ref {com_flow}。

\begin{figure}[H]
	\centering
	\includegraphics[height=.8\columnwidth]{chap04_com_flow}
	\caption {冲突检测模块流程}
	\label {com_flow}	
\end{figure}

\section{本章小结}

本章中主要介绍了软件变更冲突检测方法和其对应的模块设计与实现。
章节\ref {conflict_define}中介绍了冲突检测方法的相关定义。
章节\ref {chap_conflict}中介绍了冲突检测方法和其算法描述。
章节\ref {chap_mod}中介绍了冲突检测方法对应的工具模块其设计与实现。
\chapter{实验}
以Eclipse jdt core为例,进行实验,并分析得到的实验结果。
\section{补丁版本迁移}
\section{语义影响范围分析}
\subsection{程序间差异性分析}
\subsection{变更影响分析}
\section{兼容性分析}
\section{结论}
\chapter{结论}
\section{工作总结}

首先,本文对软件补丁兼容性检测问题和其应用场景进行了介绍。本文主要研究在软件演化的背景下,不同软件版本的补丁能否共享,即如何确定补丁对其他软件版本的兼容性。

其次,本文对该问题进行了深入的分析和讨论,发现该问题可以归结为多次变更间的冲突检测问题,并提出了一套软件补丁兼容性检测方法用于解决该问题,该检测方法包括以下部分:
\begin{itemize}
	\item 软件变更影响域分析,该分析过程主要用于获取变更的语义影响域(即变更影响域),分为两个子过程:
	\begin{itemize}
		\item 程序间语法差异性分析:该分析过程用于获取不同版本代码间的变更集合,该变更集合描述了程序间的语法结构上的差异性。
		\item 变更语义影响分析:根据差异性分析过程找到的变更集合,分析其变更影响域并输出,该变更影响域描述了软件系统中直接或间接受到该变更集合影响的语法结构集合。
	\end{itemize}
	\item 软件变更冲突检测:该冲突检测方法根据得到的不同变更影响域,找到变更影响域之间的重叠,该重叠位置即变更间可能存在冲突的位置。最后冲突的确定需要人工分析的辅助。
\end{itemize}

其中变更影响域分析的两个子过程可以自由替换为符合要求的相应算法实现,以提高检测方法的实用性。而冲突检测过程提出了一种较简单的自动冲突分析算法并进行了实现,更精确的分析结果目前需要人工分析的辅助,为此检测方法中需要变更语义影响分析过程提供相应的支持,以便追溯影响的来源。

最后,本文对该套检测方法给出了具体的工具设计和实现方案,并在Eclipse JDT Core项目上进行了测试,结果表明该工具是确实可用且有效的。它确实能够挖掘出补丁间的语义冲突并向用户进行报告。

综上所述,本文的主要贡献在于对软件补丁兼容性检测的问题进行了分析,并提出了一套补丁兼容性检测方法和其相应的兼容性检测工具实现。根据该工具在中型项目Eclipse JDT Core的八个不同版本上的实验结果,该检测工具对于工业界实际项目来说是可用的、正确性。

\section{未来工作}

对于本文中所提到的软件补丁兼容性检测方法和其工具实现而言,可能的未来工作方向包括:
\begin{itemize}
	\item 更换兼容性检测工具中影响域分析模块所使用的分析工具,并进行进一步的实验,讨论兼容性检测工具的精度与其所采用的具体工具之间的关系。
	\item 将兼容性检测工具对更多的工业界实际项目进行实验,进一步探讨其实用性。
\end{itemize}



%%% 其它部分
\backmatter


% 参考文献
\bibliographystyle{thubib}
\bibliography{ref/refs}


% 致谢
\begin{ack}
  首先要感谢贺飞老师对我的悉心指导,他对我的毕设工作和毕业论文的写作给出了许多宝贵的意见,并且不厌其烦的帮助我对论文的内容编排和组织形式进行修改。我被其认真负责的作风受到了深深的感染。
  
  其次要感谢实验室的同学和师兄等的帮助,如郭心睿、周旻、刘盛鹏等,他们在我写论文的过程中提供了很多帮助,让我铭感于心。
  
  最后要感谢\thuthesis,帮助我顺利完成了本文的写作。
\end{ack}

% 附录
\begin{appendix}
	\input{data/appendix01}
\end{appendix}

% 个人简历
\begin{resume}

  \resumeitem{个人简历}

  1990 年 04 月 03 日出生于 四川 省 绵阳 市。
  
  2008 年 9 月考入 北京理工 大学 软件学院 软件工程 专业,2012 年 7 月本科毕业并获得 工学 学士学位。
  
  2012 年 9 月进入 清华 大学 软件学院 攻读 工程硕士 学位至今。


\end{resume}

\end{document}