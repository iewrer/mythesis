\chapter{结论}
\section{工作总结}

首先,本文提出了软件补丁兼容性检测的问题,并给出了详细的形式化定义。

为了解决这样一个问题,本文通过对该问题进行深入的分析和讨论,提出了一套通用的补丁兼容性分析解决方案,包括以下部分:
\begin{itemize}
	\item 补丁应用
	\item 语义影响域分析
	\begin{itemize}
		\item 程序间差异性分析
		\item 变更影响分析
	\end{itemize}
	\item 冲突分析
\end{itemize}

其中补丁应用过程中的$merge$函数和语义影响域分析的两个子过程$diff$函数和$impact$函数可以自由使用符合要求的相应算法实现,以提高解决方案的实用性。

目前冲突分析过程提出了一种较简单的自动冲突分析算法作为$conflict$函数的实现,更精确的分析结果目前需要人工分析的辅助,为此解决方案中需要变更影响分析过程提供影响追踪函数$impact_track$来追溯语义影响的来源。

其次,本文对该套解决方案采用已有工具给出了具体的工具设计和实现方案,并对该套兼容性检测工具在Eclipse JDT Core项目上进行了测试,发现该工具是确实可用而有效的。它切实能够挖掘出补丁间的语义冲突并向用户进行报告。

可见,本文的主要成果包括:
\begin{itemize}
	\item 提出了软件补丁兼容性检测的问题。
	\item 提出了一套补丁兼容性分析的通用解决方案。
	\item 将解决方案实例化,给出了具体的工具设计与实现方案,并对中型项目Eclipse JDT Core的八个不同版本实施,论证了该解决方案对工业界实际项目的可用性。
\end{itemize}

\section{未来工作}

对于本文中所提到的补丁兼容性解决方案和其工具实现,其可能的未来工作方向包括:
\begin{itemize}
	\item 更换兼容性检测工具中影响域分析模块所使用的工具进行进一步的实验,论证该解决方案的普遍适用性,并讨论兼容性分析的精度与所采用的工具之间的关系。
	\item 将该套工具对更大型的项目进行实验。
\end{itemize}