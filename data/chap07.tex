\chapter{结论}
\section{工作总结}

如前所述,本文主要的工作是在于解决问题\ref {problem}。为此,本文提出了一套具有普遍适用性的补丁兼容性分析解决方案,该解决方案主要由三个组件构成,即:
\begin{itemize}
	\item 补丁版本迁移
	\item 语义影响分析
	\begin{itemize}
		\item 程序间差异性分析
		\item 变更影响分析
	\end{itemize}
	\item 冲突分析
\end{itemize}

其中语义影响分析组件的两个子组件可以自由替换成其他相应的算法,以提高解决方案的实用性。

目前冲突分析组件提出了一种较简单的自动冲突分析方法,更精确的分析结果目前需要人工分析的辅助,为此解决方案中需要变更影响分析组件实现影响追踪系统,以便追溯语义影响的来源。

可见,本文的主要成果包括:
\begin{itemize}
	\item 提出了一套补丁兼容性分析的通用解决方案。
	\item 利用相应的已有工具实现了语义影响分析组件,将解决方案实例化,并对中型项目Eclipse JDT Core的八个不同版本实施,论证了该解决方案对工业界实际项目的可用性。
\end{itemize}

\section{未来工作}

对于本文中所提到的补丁兼容性解决方案,其可能的未来工作方向包括:
\begin{itemize}
	\item 根据解决方案的要求,进一步整合已有工具,实现解决方案的工具化。
	\item 更换语义影响分析组件中所使用的工具并继续实验,论证该解决方案的普遍适用性,并论证兼容性分析的精度与所采用的工具之间的关系。
	\item 对更大型的项目进行实验。
\end{itemize}