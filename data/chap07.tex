\chapter{结论}
\section{工作总结}

首先,本文对软件补丁兼容性检测问题和其应用场景进行了介绍。

其次,为了解决这样一个问题,本文通过对该问题进行深入的分析和讨论,提出了一套补丁兼容性检测方法,它包括以下部分:
\begin{itemize}
	\item 软件变更影响域分析,该分析过程主要用于获取变更的语义影响域(即变更影响域),分为两个子过程:
	\begin{itemize}
		\item 程序间语法差异性分析:用于获取不同版本间代码的变更集合。
		\item 变更语义影响分析:根据找到的变更集合,分析其变更影响域并输出。
	\end{itemize}
	\item 软件变更冲突检测:该冲突检测方法根据得到的变更集合的影响域,找到变更间可能存在冲突的位置。
\end{itemize}

其中变更影响域分析的两个子过程可以自由使用符合要求的相应算法实现,以提高解决方案的实用性。

目前冲突分析过程提出了一种较简单的自动冲突检测算法并进行了实现,更精确的分析结果目前需要人工分析的辅助,为此解决方案中需要变更语义影响分析过程提供影响追踪系统来追溯影响的来源。

最后,本文对该套解决方案给出了具体的工具设计和实现方案,并对该套兼容性检测工具在Eclipse JDT Core项目上进行了测试,发现该工具是确实可用而有效的。它确实能够挖掘出补丁间的语义冲突并向用户进行报告。

可见,本文的主要贡献包括:
\begin{itemize}
	\item 对软件补丁兼容性检测的问题进行了分析。
	\item 提出了一套补丁兼容性分析的通用解决方案。
	\item 根据提出的解决方案,实现了具体的兼容性检测工具,并在中型项目Eclipse JDT Core的八个不同版本上进行了实验,论证了该解决方案对工业界实际项目的可用性。正确性和实用性。
\end{itemize}

\section{未来工作}

对于本文中所提到的补丁兼容性解决方案和其工具实现,其可能的未来工作方向包括:
\begin{itemize}
	\item 更换兼容性检测工具中影响域分析模块所使用的工具进行进一步的实验,讨论兼容性检测工具的精度与其所采用的具体工具之间的关系。
	\item 将兼容性检测工具对更多的工业界实际项目进行实验,进一步探讨其实用性。
\end{itemize}