\chapter{分析方法}
\section{问题定义}
该问题可以拆分为若干个子问题,并分别阐述并定义之:

\subsection{补丁版本迁移}
\subsection{变更的语义影响分析}
\subsection{兼容性分析}

\section{应用场景}
\section{解决方案}
介绍补丁兼容性分析方法的整体架构:补丁版本迁移、语义影响范围分析(程序差异性分析、程序变更影响分析)、补丁兼容性分析。
\section{流程}
介绍补丁兼容性分析的整体流程:【可给出一张抽象的流程图】。
\subsection{补丁版本迁移}
\subsection{语义影响范围分析}
\subsubsection{程序差异性分析}
\subsubsection{变更影响分析}
\subsection{实际流程}

	\begin{enumerate}
		\item 采用git进行版本管理。
		\item 将新版本和补丁版本两个分支进行合并,得到应用于新版本的补丁版本。
		\item 采用AST Differ生成程序间差异性文件,XML格式.
		\item 采用jpf-regression进行变更影响分析,获取三个版本间两两的影响集合,dot格式
		\item 对两个影响集合求交集,若无交集,则兼容之,若有交集,进一步采用分析算法,看是否确实不兼容
	\end{enumerate}
	
	【此时可给出一张具体的流程图】
