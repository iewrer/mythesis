\chapter{绪论}
\section{研究背景和意义}

软件维护(Software maintenance)是软件开发周期中耗时最长、开销最大的过程。随着外部环境和用户需求的不断变化,软件系统需要随之进行适应和调整,同时也需要修复在实际运行中暴露出来的问题。

补丁(Patch)就是这样一类可以用于完成修补程序漏洞、增强软件功能、改善程序性能等任务的程序。补丁在工业界中得到了广泛的实际应用,是软件维护过程的重要组成部分。

补丁一般是通过Diff工具而产生的文本数据,用以表示以行为单位的程序间差异性。一般而言,Diff工具属于程序差别分析工具中的一种,单纯进行文本的差异性比较,并给出行级别的差异性信息。然而这样产生出来的差异性信息(即patch),只包含了文本差异,而失去了语法、语义等差异信息。

然而,补丁程序仍然具有一定的局限性,它一般只针对某个专门软件版本而开发,对于软件演进过程中获得的新版本而言,我们无法确定补丁程序是否也能适用于新版本的程序。然而补丁程序的应用一般都具有特定的目的,例如功能升级、漏洞修补等,新版本的程序中很可能仍然需要补丁程序的应用来完善自身。



\section{本文所要解决的问题与主要工作}

软件系统总是处于不断演进的过程中的。补丁(patch)是一种常见的软件系统版本演进的方法,常用于修补漏洞和缺陷、改进性能、升级等。在实际应用中,补丁往往针对某一个特定版本的代码而设计,然而现代软件系统往往更新换代较快,因而面临着将补丁应用于其他版本的代码时可能失效的问题。为了解决这个问题,我们为此提出了进行补丁兼容性分析的想法。

为此,我们考虑这样一个问题,针对软件系统s,假设已有有版本v1和v2,其中补丁p1是针对版本v1而设计,打上后会演进到版本v3,问p1是否能够应用于版本v2,并保证不会发生冲突。从问题中可以发现,p1 = diff(v1,v3),设p2 = diff(v1,v2),那么该问题就可以规约成看两个补丁p1和p2间是否会发生冲突?

对该问题而言,其答案可以分为多个层面来回答:

\begin{enumerate}
	\item
	语法:从语法角度出发,则该问题主要关注的是在应用过程中是否会造成语法结构上的错误,如果能够将补丁p1成功应用于程序版本v2,则认为该补丁是可以兼容于新版本v2的。
	\item
	语义:从语义角度出发,则单纯的语法兼容并不能够完全解答这个问题。某行修改可能会影响多处源代码,从而导致程序的行为发生变化。因而从语义层面而言,我们需要保证补丁p1对程序版本v2造成的语义影响不会波及到从版本v1演进到v2时所造成的语义变化。
\end{enumerate}

语法层面的回答很容易就能给出,现有的版本控制系统等都能在一定程度上给出相应的答案。而语义层面的答案,就目前所知尚未有这方面的工作。

因而本文将着重从语义层面尝试去解决这个问题。该问题可以较形式化地描述如下:

\begin{problem}
	具体定义
\end{problem}

目前而言,本文的主要工作包括以下几个部分:

\begin{enumerate}
	\item
	补丁版本迁移

补丁s1本来是适用于程序版本p1的,如果想要适用于程序p2,可能需要进行一定的版本迁移工作。

该部分工作可以从语法层面给出兼容性的答案,并进行补丁兼容性的语法结构修正。

	\item
	语义影响范围分析

本文主要采用程序间差异分析、变更影响分析等手段来对界定补丁s1对程序p2所造成的语义影响范围。

所谓的语义影响范围(semantic impact area)可以具体定义如下,其中structure意为程序语法结构,可以采用不同级别的程序语法结构(如statement、basic block等)进行分析,来获得不同粒度的影响范围:
	\begin{enumerate}
		\item
		change(s) = { structurei | structurei属于p,并且structurei 发生了变更},该集合可以采用程序间差异分析获得。
		
		\item
		impact(s) = {structurei| structurei属于p,并且structurei 受到change(s)的影响},该集合可以采用变更影响分析获得。
		
	\end{enumerate}
最后得到的impact(s)即为我们所需的变更的语义影响范围。

	\item
	补丁兼容性分析

在有了语义影响集合impact(s)之后,可以进行具体的兼容性分析工作。
\end{enumerate}

\section{本文组织结构}

本文主要包括七个章节,第一章是绪论,介绍本文的研究背景和主要工作;第二章主要介绍与本文所述内容相关的国内外的工作;第三章主要介绍了补丁兼容性分析的具体方法;第四章主要介绍了如何将各阶段的不同分析方法进行整合,形成具体的流程;第五章详细介绍了具体的补丁兼容性分析算法;第六章介绍了实验过程和结果;第七章主要对本文的工作进行了总结,并提出了进一步的工作方向。


