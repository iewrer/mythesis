\chapter{相关工作}
\section{软件演进}
\section{程序间差异分析}
\section{程序变更影响分析}
\section{相关工具}
	主要介绍本文中采用到的相关工具:
	
	\begin{itemize}
		\item git

git是一个分布式的版本控制系统,最初由 Linus Torvalds在2005年为Linux内核而开发,现在已经成为最流行的版本控制系统。

与CVS和SVN等集中式的C/S版本控制系统不同,git是分布式的版本库,每个本地的git工作目录都包含了完整的历史数据和版本追踪能力,无需网络连接或服务器端。
      

本文中主要考虑以git作为版本管理系统的应用场景,类似于GitHub,假定为项目开发了new version和patch version两个不同的分支,并使用git的分支合并策略实现补丁的版本迁移过程。
		\item beyond compare
		
Beyond Compare 是一款内容比较工具,可以用于文件、目录、压缩包的比较,横跨Windows、Mac OS X、Linux三大操作系统,可用作版本控制系统的文本比较和合并工具,例如git。
      
本文中主要采用其作为git的文本比较和合并工具,用于解决补丁版本迁移时的冲突(conflict)问题。

		\item AST Differ
		\item jpf-regression 

	\end{itemize}