\ctitle{基于影响域分析的软件补丁兼容性检测}
\cdegree{工程硕士}
\cdepartment[软件]{软件学院}
\cmajor{软件工程}
\cauthor{潘晓梦} 
\csupervisor{贺飞副教授}

\edegree{Master of Software Engineering} 
\emajor{Software Engineering} 
\eauthor{Pan Xiaomeng} 
\esupervisor{Professor He Fei}

% 定义中英文摘要和关键字
\begin{cabstract} 
	众所周知,软件维护是软件开发周期中耗时最长,也是开销最大的过程。软件维护过程中,经常伴随着软件版本不断演进的过程。在实际应用中,补丁程序是最常见的用于软件版本演进的一类程序,它能够实现软件功能增强、漏洞修复等。
	
	补丁程序的局限性在于,它一般都是针对于某个专门版本的代码而设计,通常无法正确的在其他版本代码上使用。然而,在实际开发过程中工程师经常能够遇到这样的问题,如果对其他版本代码重新开发专门的补丁程序,其资源消耗较多。
	
	为此,本文中提出了一套基于语义影响域分析的软件补丁兼容性检测方案,并且实现了相应的工具。语义影响域分析可以挖掘变更对于软件代码的语义影响,根据该分析过程中得到的语义影响域,我们能够进行补丁间的语义冲突检测。通过在Eclipse JDT Core上的实验,我们发现该工具确实能够发现软件补丁的兼容性问题,本文中所提出的解决方案是可用的、可靠的。
\end{cabstract}
\ckeywords {软件维护,语义影响域,补丁,兼容性检测}
\begin{eabstract} 
		As well known,software maintenance is the longest and most cost activity in the software developing cycle.In software maintenance,the software system sometimes got evolved.In pratical application,patch is the most common program which used for software evolving,it can enhance it's functionality,fix it's bug and so on.
		
		The limitation of patch program is that it's usually designed for a specific version of software so that it cannnot be applied to annother version directly.However,it is a common question happended during the software development and developing another patch program for this specific version seems too costly.
		
		Under the above circumstance,we propose a solution called software patch compatibility checking based on semantic impacted area analysis for this problem and develop a corresponding tool.The semantic impacted aread analysis can mines semantic impact introduced by software change from the code.And using it's result we could check if there is a semantic confltct issue between the patches.After experimenting on the Eclipse JDT Core project,the tool is proven to having the ability to find out compatibility issues between the patches and this solution is applicable and sound.  
		
		
%		The software system will continue evolving in its whole lift time,and patch is a common way to accomplish the task to update or fix bug and so on.As the software evolving is rapid, how to identify whether a patch for a specific version of software is applicable for an newer version of the same software.We here present a so called patch compatibility analysis to try to fix this problem.It consists of three kind of analysis—program differencing which is used to output the structural difference of the two versions,change impact analysis which is used to analysis the impact of every change including dependency and impacted expressions/statements and so on.And at last,the conflict analysis used to determine whether the two patches’ impact are confictual.
\end{eabstract}


\ekeywords{software maintenance,semantic impacted area,patch,compatibility checking}