\chapter{方法实现}
本章中将主要介绍如何将补丁版本迁移、语义影响范围分析、兼容性分析等过程具体实现之,并组合成整体流程。

\section{补丁版本迁移}

如前所述,补丁版本迁移主要用于解决问题\ref {patch_reversion}。我们在实现该过程时,主要采用了git工具和Beyond Compare工具来实现版本合并和冲突解决的过程。

我们下面将利用git的分支功能进行版本合并。首先介绍一些前提假设:

\begin{enumerate}
	\item 假设git的工作目录如图所示。

	\item 假设补丁$p_1$为old.patch文件,并且采用git diff或其他Unix diff命令生成。
\end{enumerate}

下面将具体介绍整个版本合并的过程。


首先在工作目录下将git切换到主分支master。然后在主分支master中提交代码版本$v_1$。

\begin{lstlisting} [style=BashInputStyle]
git checkout master
git add .
git commit -a -m "old version committed"
\end{lstlisting}

然后新建并切换到分支new中,并提交代码版本$v_2$。
\begin{lstlisting} [style=BashInputStyle]
git checkout -b new
git add .
git commit -a -m "new version committed"
\end{lstlisting}

再次新建并切换到分支patched\_old,然后将补丁$p_1$应用到版本$v_1$,获得旧版本应用补丁后的代码,其版本为$v_3$,并提交代码版本$v_3$。此时我们所应用的补丁$p_1$是专为版本$v_1$设计,所以应用时不会出现问题。

\begin{lstlisting} [style=BashInputStyle]
git checkout -b patch
git apply ../patch/old.patch
git add .
git commit -a -m "patched version from old version committed"
\end{lstlisting}

然后再切换回分支new,将分支patch合并入分支new,获得新版本应用补丁后的代码,其版本为$v_4$,并使用Beyond Compare工具解决可能出现的冲突问题。将冲突解决完毕后,再提交版本$v_4$。

\begin{lstlisting} [style=BashInputStyle]
git checkout new
git merge patch
git mergetool
git commit -a -m "patched version from new version committed"
\end{lstlisting}

如果确实有冲突,那么git mergetool命令会调用第三方的可视化合并工具并引导你去解决冲突。这里我们采用的合并工具即为Beyond Compare,它会展开一个可视化的界面,并给出冲突位置的提示,方便进行人工选择、合并。

整个过程可以参考图示。

\section{语义影响范围分析}
\subsection{程序间差异分析}
\subsection{变更影响分析}
\section{兼容性分析}